
\section{La Calculadora}

\subsection{Uso desde Línea de Comandos}
\begin{verbatim}
[~/src/PL/rexical/sample(master)]$ racc --help
Usage: racc [options] <input>
    -o, --output-file=PATH           output file name [<input>.tab.rb]
    -t, --debug                      Outputs debugging parser.
    -g                               Equivalent to -t (obsolete).
    -v, --verbose                    Creates <filename>.output log file.
    -O, --log-file=PATH              Log file name [<input>.output]
    -e, --executable [RUBYPATH]      Makes executable parser.
    -E, --embedded                   Embeds Racc runtime in output.
        --line-convert-all           Converts line numbers of user codes.
    -l, --no-line-convert            Never convert line numbers.
    -a, --no-omit-actions            Never omit actions.
        --superclass=CLASSNAME       Uses CLASSNAME instead of Racc::Parser.
        --runtime=FEATURE            Uses FEATURE instead of 'racc/parser'
    -C, --check-only                 Checks syntax and quit immediately.
    -S, --output-status              Outputs internal status time to time.
    -P                               Enables generator profile
    -D flags                         Flags for Racc debugging (do not use).
        --version                    Prints version and quit.
        --runtime-version            Prints runtime version and quit.
        --copyright                  Prints copyright and quit.
        --help                       Prints this message and quit.

\end{verbatim}

\begin{verbatim}
[~/Dropbox/src/PL/rexical/sample(master)]$ cat -n Rakefile 
     1  task :default => %W{racc rex} do
     2    sh "ruby calc3.tab.rb"
     3  end
     4  
     5  task :racc do
     6    sh "racc calc3.racc"
     7  end
     8  
     9  task :rex do
    10    sh "rex calc3.rex"
    11  end
\end{verbatim}

\subsection{Análisis Léxico con {\tt rexical}}
\begin{verbatim}
[~/Dropbox/src/PL/rexical/sample(master)]$ cat -n calc3.rex
     1  #
     2  # calc3.rex
     3  # lexical scanner definition for rex
     4  #
     5  
     6  class Calculator3
     7  macro
     8    BLANK         \s+
     9    DIGIT         \d+
    10  rule
    11    {BLANK}
    12    {DIGIT}       { [:NUMBER, text.to_i] }
    13    .|\n          { [text, text] }
    14  inner
    15  end
\end{verbatim}

\subsection{Análisis Sintáctico}
\begin{verbatim}
[~/Dropbox/src/PL/rexical/sample(master)]$ cat -n calc3.racc 
     1  #
     2  # A simple calculator, version 3.
     3  #
     4  
     5  class Calculator3
     6    prechigh
     7      nonassoc UMINUS
     8      left '*' '/'
     9      left '+' '-'
    10    preclow
    11    options no_result_var
    12  rule
    13    target  : exp
    14            | /* none */ { 0 }
    15  
    16    exp     : exp '+' exp { val[0] + val[2] }
    17            | exp '-' exp { val[0] - val[2] }
    18            | exp '*' exp { val[0] * val[2] }
    19            | exp '/' exp { val[0] / val[2] }
    20            | '(' exp ')' { val[1] }
    21            | '-' NUMBER  =UMINUS { -(val[1]) }
    22            | NUMBER
    23  end
    24  
    25  ---- header ----
    26  #
    27  # generated by racc
    28  #
    29  require 'calc3.rex'
    30  
    31  ---- inner ----
    32  
    33  ---- footer ----
    34  
    35  puts 'sample calc'
    36  puts '"q" to quit.'
    37  calc = Calculator3.new
    38  while true
    39    print '>>> '; $stdout.flush
    40    str = $stdin.gets.strip
    41    break if /q/i === str
    42    begin
    43      p calc.scan_str(str)
    44    rescue ParseError
    45      puts 'parse error'
    46    end
    47  end
\end{verbatim}


\parrafo{Precedencias}
\verb'right' is yacc's \verb|%right|, \verb'left' is yacc's \verb|%left|.

\verb'= SYMBOL'  means yacc's \verb|%prec SYMBOL|:
\begin{verbatim}
prechigh
  nonassoc '++'
  left     '*' '/'
  left     '+' '-'
  right    '='
preclow
\end{verbatim}


\begin{verbatim}
rule
  exp: exp '*' exp
     | exp '-' exp
     | '-' exp       =UMINUS   # equals to "%prec UMINUS"
         :
         :
\end{verbatim}

\parrafo{Atributos}
You can use following special local variables in action.

\begin{enumerate}
\item
\begin{verbatim}
result ($$)
\end{verbatim}
The value of left-hand side (lhs). A default value is \verb|val[0]|.

\item
\begin{verbatim}
val ($1,$2,$3...)
\end{verbatim}
An array of value of right-hand side (rhs).

\item
\begin{verbatim}
_values (...$-2,$-1,$0)
\end{verbatim}
A stack of values. DO NOT MODIFY this stack unless you know what you are doing.
\end{enumerate}

\parrafo{Declaring Tokens}
By declaring tokens, you can avoid bugs. 

\begin{verbatim}
token NUM ID IF
\end{verbatim}

\parrafo{Opciones}

You can write options for racc command in your racc file.

\begin{verbatim}
options OPTION OPTION ...
\end{verbatim}
Options are:

\begin{enumerate}
\item
\begin{verbatim}
omit_action_call
\end{verbatim}
omit empty action call or not.

\item
\begin{verbatim}
result_var
\end{verbatim}
use/does not use local variable "result"
\end{enumerate}

You can use \verb'no_' prefix to invert its meanings.


\parrafo{User Code Block}

"User Code Block" is a Ruby source code which is copied to output. There are three user code block, 
"header" 
"inner" 
and "
footer".

Format of user code is like this:

\begin{verbatim}
---- header
  ruby statement
  ruby statement
  ruby statement

---- inner
  ruby statement
     :
     :
\end{verbatim}
If four \verb'-' exist on line head, 
racc treat it as beginning of user code block. 
A name of user code must be one word.

\section{Véase También}

\begin{itemize}
\item
\htmladdnormallink{Racc en GitHub}{https://github.com/tenderlove/racc}
\item
\item
\htmladdnormallink{Racc User's Manual}{http://i.loveruby.net/en/projects/racc/doc/}
\item
\htmladdnormallink{Martin Fowler Hello Racc}{http://martinfowler.com/bliki/HelloRacc.html}
\item
\htmladdnormallink{Rexical en GitHub}{https://github.com/tenderlove/rexical}
\end{itemize}
