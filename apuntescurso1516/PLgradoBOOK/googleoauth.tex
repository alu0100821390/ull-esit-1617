\section{Introduction to OAuth}
OAuth 2 is rapidly becoming a preferred authorization protocol, and is used by major service providers such as Google, Facebook and Github. 

\parrafo{Valet Key for the Web}

Many luxury cars come with a \blue{valet key}. 
It is a special key \blue{you}
 give the
\blue{parking attendant} and unlike your regular key, will only allow the 
\blue{car}
to be driven a short distance while \red{blocking access to the trunk and the
onboard cell phone}. 

Regardless of the restrictions the valet key imposes,
the idea is very clever. \red{You give someone limited access to your car
with a special key}, while using another key to unlock everything else.

As the web grows, more and more sites rely on distributed services and
cloud computing: 
\begin{itemize}
\item
a photo lab printing your Flickr photos, 
\item
a social network
using your Google address book to look for friends, or 
\item
a third-party
application utilizing APIs from multiple services.
\end{itemize}

The problem is, in order for these applications to access user data
on other sites, they ask for usernames and passwords. \red{Not only does
this require exposing user passwords to someone else – often the same
passwords used for online banking and other sites – it also provides
these application \underline{unlimited access} to do as they wish}. They can do
anything, including changing the passwords and lock users out.

\OAuth{} provides a method for \blue{users} (you) 
to grant \blue{third-party} (parking attendant) access to
their \blue{resources} (your luxury car)
without sharing their \blue{passwords} (the key of your car). 
It also provides a way
to \underline{grant limited access} (in scope, duration, etc. the equivalent of
not having access to the trunk or the onboard cell phone).

For example, 
\begin{itemize}
\item
a \blue{web user (resource owner)} can grant a 
\item
\blue{printing service}
(client) 
\item
access to her \blue{private photos} (partial resource) 
\item
stored at a photo sharing service
(server), 
\item
without sharing her username and password with the printing
service.  
\end{itemize}
Instead, she authenticates directly with the photo sharing
service which issues the printing service 
delegation-specific credentials.

In OAuth, the \blue{client} requests access to resources controlled
by the \blue{resource owner} and hosted by the \blue{resource server}, 
and {\it is
issued a different set of credentials than those of the resource
owner}.

Instead of using the resource owner's credentials to access protected
resources, the \blue{client} obtains an \red{access token} 
-- {\it a string denoting a
specific scope, lifetime, and other access attributes}.  

\red{Access tokens}
are issued to third-party clients by an \blue{authorization server} with the
approval of the \blue{resource owner}.  

The client uses the \red{access token} to
access the \blue{protected resources} hosted by the \blue{resource server}.

\begin{rawhtml}
<img src="oauth-print-service-access-token-flow.gif" />
\end{rawhtml}

\begin{enumerate}
\item
\htmladdnormallink{The OAuth 2.0 Authorization Framework}{http://tools.ietf.org/html/rfc6749} proposed standard document
\end{enumerate}


\parrafo{Roles in OAuth}
   OAuth defines four roles:

\begin{enumerate}
\item \cei{resource owner}

      An entity capable of granting access to a protected resource.
      When the resource owner is a person, it is referred to as an
      \cei{end-user}.

\item \cei{resource server}

      The server hosting the protected resources, capable of accepting
      and responding to protected resource requests using access tokens.

\item \cei{client}

      An application making protected resource requests on behalf of the
      resource owner and with its authorization.  

The term "\blue{client}" does
      not imply any particular implementation characteristics (e.g.,
      whether the application executes on a server, a desktop, or other
      devices).

\item \cei{authorization server}

      The server issuing access tokens to the client after successfully
      authenticating the resource owner and obtaining authorization.
\end{enumerate}

\begin{rawhtml}
<img src="oauth-players.jpg" />
\end{rawhtml}

The authorization server
may be the same server as the resource server or a separate entity.
A single authorization server may issue access tokens accepted by
multiple resource servers.

\parrafo{Véase}
\begin{itemize}
\item
\htmladdnormallink{Nacho Coloma: Our love-hate relationship with OAuth}{https://plus.google.com/+NachoColoma/posts/SQ7F1ncvSkn}
\end{itemize}

\section{Google Developers Console}

\subsection{Managing projects and applications}

A project consists of 
\begin{enumerate}
\item
a set of applications, 
\item
along with activated APIs,
\item
Google Cloud resources, and 
\item
the team and billing information associated
with those resources.
\end{enumerate}
\red{Credentials such as API keys are specific to an application rather than to
a project}. 

However, all applications within a given project use the same
branding information (logo, email address, etc.) on their user consent
screen, as described in 
\htmladdnormallink{Setting up OAuth 2.0.}{https://developers.google.com/console/help/new/\#generatingoauth2}

Applications within a
project also share 
\begin{enumerate}
\item
activated APIs, 
\item
permissions, and 
\item
billing information.
\end{enumerate}

%\subsubsection{Creating and deleting projects}
%
%You can use one Google Developers Console project to manage all of your
%applications, or you can create a different project for each one. 
%
%In
%deciding whether to create a new project for a given application,
%consider whether 
%\begin{itemize}
%\item
%you're collaborating with a different set of people,
%\item
%want to track usage differently, or 
%\item
%would set different traffic controls
%for each application. 
%\end{itemize}
%If so, segregating applications by project might
%make sense. You can create multiple projects, 
%\red{but remember that you
%cannot use multiple projects to try to exceed the limits for API usage
%by your application}.
%
%\subsubsection{Creating a project and registering an application}
%
%
%\parrafo{Create a project}
%
%To \red{create a project}, do the following:
%
%\begin{itemize}
%\item
%Visit the 
%\htmladdnormallink{Google Developers Console}{https://console.developers.google.com/}
%and 
%\item
%select \red{Create Project}. 
%\begin{itemize}
%\item
%Enter a \red{name} and a \red{project ID}, or accept the
%defaults, and select \red{Create}.
%\end{itemize}
%\end{itemize}
%
%\parrafo{Register a new application}
%To \red{register a new application}, do the following:
%
%\begin{itemize}
%\item
%Go to the 
%\htmladdnormallink{Google Developers Console }{https://console.developers.google.com/}
%\item
%Select a \red{project}, or create a new one.
%\item
%In the sidebar on the left, select \verb|APIs & auth|. 
%\item
%In the displayed list
%of APIs, \red{make sure all the APIs you are using show a status of} \green{ON}.
%\item
%In the sidebar on the left, select Registered apps.
%At the top of the page, select Register App.
%\item
%Fill out the form and select Register.
%\end{itemize}
%
%\subsubsection{Deleting a project}
%
%Deleting a project from the Developers Console releases all resources
%used within the project. Only project owners can delete projects, so
%only project owners will see the delete (and undelete) options within
%the Developers Console.
%
%To delete a project, do the following:
%
%\begin{itemize}
%\item
%Disable billing on the project.
%\item
%Mark the project for deletion in the Developers Console. 
%\item
%At this point, Google emails all project members to inform them of the pending project deletion.
%\item
%If you change your mind, undelete the project. 
%\item
%Undeleting a project is
%possible for seven days, counting from the time you marked the project
%for deletion. 
%\item
%Undeleting returns the project to the state it was in
%prior to deletion.
%\item
%The time it
%takes to delete a project might vary depending on the number or kind of
%services in use within the project.
%\end{itemize}
%
%\subsubsection{Project ID and project number}
%
%There are two ways to identify your project:
%
%\begin{itemize}
%\item
%The \cei{project ID} is an identifier that you select when you create the
%project, and it is used only within the Developers Console. 
%
%\red{A project ID} 
%is unique and cannot be changed, so choose an ID that you'll be
%comfortable using for the lifetime of the project.
%
%To determine your project ID, visit the 
%\htmladdnormallink{Google Developers Console}{https://console.developers.google.com/}
% Find your project in the list on the main landing page. The project ID appears in the second column of the table.
%
%\item
%The \cei{project number} is assigned by the 
%\htmladdnormallink{Google Developers Console }{https://console.developers.google.com/}.
%
%To determine your project number, visit the 
%\htmladdnormallink{Google Developers Console }{https://console.developers.google.com/}.
% Select
%your project in the list on the main landing page. The project number
%appears at the top of the project's overview page.
%\end{itemize}
%
\subsubsection{Managing project members}

If you create a project, you have owner-level permissions and can grant
owner-level permissions to other project members. Those with owner-level
permissions are \red{project owners}.

Only project owners can add and remove other project members and edit
their permission levels. Project owners can share a project with an
email address that represents a group, but every project must have at
least one project member that is an individual, not a group.

To manage project members, do the following:

\begin{enumerate}
\item
Visit the 
\htmladdnormallink{Google Developers Console }{https://console.developers.google.com/}
\item
Select a project, or create a new one.
\item
In the sidebar on the left, select \verb|Permissions|.
\item
To add a team member or group, select \verb|Add Member|. 
\begin{itemize}
\item
You must provide an
\red{email address} that is associated with a Google account. 
\item
If the email
address belongs to an individual, an invitation flow is triggered,
and the new project member must accept the invitation before they can
access the project. 
\item
If the email address belongs to a group, the group
is added right away, with no invitation step.
\end{itemize}
\item
To change the permission setting for a project member, click the dropdown
box in the Permission column and select a new permission level. The new
permission level is saved automatically.
\item
To delete a project member, click the trash icon to the right of the
project member's permission setting.
\end{enumerate}

%\subsection{Activating or deactivating APIs}
%
%Activating an API associates that API with the current project and 
%\begin{enumerate}
%\item
%adds
%monitoring pages for that API, 
%\item
%enables billing for that API, and 
%\item
%adds
%any custom information to the Developers Console for that API.
%\end{enumerate}
%
%Some APIs are in preview mode, meaning that they require special signup
%and whitelisting before you can try them out.
%
%To activate an API for your project, do the following:
%\begin{enumerate}
%\item
%Go to the 
%\htmladdnormallink{Google Developers Console }{https://console.developers.google.com/}
%\item
%Select a project.
%\item
%In the sidebar on the left, select \verb|APIs & auth|.
%\item
%In the displayed list of available APIs, find the one you want to
%activate, and set its status to \green{ON}.
%\end{enumerate}
%
\subsection{Keys, access, security, and identity}

Each request to an API that is represented in the Google Developers
Console must include a \red{unique identifier}. 

Unique identifiers enable
the Developers Console to tie requests to specific projects in order to
\begin{itemize}
\item
monitor traffic, 
\item
enforce quotas, and 
\item
handle billing.
\end{itemize}

Google supports two mechanisms for creating unique identifiers:

\begin{enumerate}
\item
\htmladdnormallink{OAuth 2.0 client IDs}{https://developers.google.com/console/help/new/\#generatingoauth2}

For applications that use the OAuth 2.0 protocol to call Google APIs,
you can use an OAuth 2.0 client ID to generate an access token. The
token contains a unique identifier.

\item
\htmladdnormallink{API keys}{https://developers.google.com/console/help/new/\#generatingdevkeys}

An API key (either a server key or a browser key) is a unique identifier
that you generate using the Developers Console. Using an API key does
not require user action or consent. \red{API keys do not grant access to any
account information, and are not used for authorization}.
\end{enumerate}

Use an API key when your application is running on a server and accessing
one of the following kinds of data:

\begin{itemize}
\item
Data that the data owner has identified as public, such as a public
calendar or blog.
\item
Data that is owned by a Google service such as Google Maps or Google
Translate. (Access limitations may apply.)
\item
If you are only calling APIs that do not require user data, such as the
Google Custom Search API, then API keys might be simpler to use than
OAuth 2.0 access tokens. However, if your application already uses an
OAuth 2.0 access token, then there is no need to generate an API key as
well. Google ignores passed API keys if a passed OAuth 2.0 access token
is already associated with the corresponding project.
\end{itemize}

%\subsection{Setting up OAuth 2.0}
%
%To use OAuth 2.0 in your application, you need an OAuth 2.0 client
%ID, which your application uses when requesting an OAuth 2.0 access
%token. When you register an application in the Google Developers Console,
%you can generate an OAuth 2.0 client ID.
%
%To find your application's client ID and client secret, and set a redirect
%URI, expand the OAuth 2.0 Client ID section.
%
%To deactivate the client ID, delete the application from the Developers
%Console.
%
%\subsubsection{User consent}
%
%When you use OAuth 2.0 for authentication, your users are authenticated
%after they agree to terms presented to them on a user consent screen.
%
%To set up your project's consent screen, do the following:
%
%\begin{enumerate}
%\item
%Go to the Google Developers Console.
%\item
%Select a project.
%\item
%In the sidebar on the left, select \verb|APIs & auth|. 
%\item
%In the displayed list of APIs, make sure all the APIs you are using show a status of \green{ON}.
%\item
%In the sidebar on the left, select \red{Consent screen}.
%\item
%Fill in the form and select \red{Save}.
%
%The consent screen settings within the Developers Console \red{are set
%at the project level}, so the information that you specify on the Consent
%screen page applies to every application within the project. 
%
%However,
%\red{applications within the same project can set different scopes in their
%authentication requests}. This means that each application within a
%project can show different information for the \verb"This app would like to:"
%section of the consent screen.
%
%\subsection{Service accounts}
%
%If your application employs server-to-server interactions such as those
%between a web application and Google Cloud Storage, then you need a
%private key and other service-account credentials. \red{To generate these
%credentials}, or to view the email address and public keys that you've
%already generated, do the following:
%
%\begin{enumerate}
%\item
%View the application:
%\begin{enumerate}
%\item
%Go to the 
%\htmladdnormallink{Google Developers Console }{https://console.developers.google.com/}
%\item
%Select a project.
%\item
%In the sidebar on the left, select \verb|APIs & auth|. 
%\item
%In the displayed list of APIs, make sure all the APIs you are using show
%a status of \green{ON}.
%\item
%In the sidebar on the left, select \red{Credentials}.
%\end{enumerate}
%\item
%To set up a service account, 
%\begin{enumerate}
%\item
%select \red{Create New Client ID}. 
%\item
%Specify
%that your application type is service account, and then 
%\item
%select Create
%Client ID. 
%\item
%A dialog box appears; to proceed, select Okay, got it. 
%\item
%(If
%you already have a service account, you can add a new key by selecting
%Generate new key beneath the existing service-account credentials. 
%A
%dialog box appears; to proceed, select Okay, got it.)
%\end{enumerate}
%\end{enumerate}
%
%Your application needs the private key when requesting an OAuth 2.0
%access token in server-to-server interactions. Google does not keep a
%copy of this private key, and this screen is the only place to obtain
%this particular private key. When you click Download private key, the
%\verb|PKCS #12|-formatted private key is downloaded to your local machine. As
%the screen indicates, you must securely store this key.
%
%The name of the downloaded private key is the key's thumbprint. When
%inspecting the key on your computer, or using the key in your application,
%you need to provide the password notasecret. Note that while the password
%for all Google-issued private keys is the same (notasecret), each key
%is cryptographically unique.
%
%You can generate multiple public-private key pairs for a single service
%account. This makes it easier to update credentials or roll them over
%without application downtime. However, you cannot delete a key pair if
%it is the only one created for that service account.
%
%Use the email address when granting the service account access to
%supported Google APIs.
%
%For more details, see the OAuth 2.0 Service Accounts documentation.
%
%Reminder: When you use a service account, you are subject to the Terms
%of Service for each product, both as an end user and as a developer.
%

\section{OmniAuth gem: Standardized Multi-Provider Authentication for Ruby}

\begin{itemize}
\item
\htmladdnormallink{OmniAuth}{http://intridea.github.io/omniauth/}
\end{itemize}

OmniAuth is a library that standardizes multi-provider authentication
for web applications. Any developer can create \cei{strategies} for
OmniAuth that can
authenticate users via disparate systems. 

OmniAuth strategies have been created for everything from Facebook to LDAP.

To use OmniAuth, you need only 
\begin{enumerate}
\item
to redirect users to \verb|/auth/:provider|,
where \verb|:provider| is the name of the strategy 
(for example, \verb|developer| or \verb|twitter|). 

\item
From there, OmniAuth will take over and take the
user through the necessary steps to authenticate them with the chosen strategy.

\item
Once the user has authenticated, 
OmniAuth sets a special hash called the \cei{Authentication Hash} 
on the Rack environment of a request to \verb|/auth/:provider/callback|. 

\item
This hash
contains as much information about the user as OmniAuth was able to
glean from the utilized strategy. 

\item
\red{You should set up an endpoint in your
application that matches to the callback URL and then
performs whatever steps are necessary for your application}. 
\end{enumerate}

\parrafo{Getting Started}

To use OmniAuth in a project with a Gemfile, just add each of the
\blue{strategies} you want to use individually:

\begin{verbatim}
gem 'omniauth-github'
gem 'omniauth-openid'
\end{verbatim}

Now you can use the \verb|OmniAuth::Builder| 
Rack middleware \blue{to build up your
list of OmniAuth strategies for use in your application}:

Para saber mas sobre Rack y sobre Middlewares Rack,
véanse las secciones 
\begin{itemize}
\item
Rack en \ref{chapter:rack},
\item
{\it Middleware y la Clase Rack::Builder} en \ref{section:middleware} 
y
\item
{\it La Estructura de una Aplicación Rack}
en
\ref{section:estructuradeunaapprack}
\end{itemize}

\begin{verbatim}
use OmniAuth::Builder do
  provider:github, ENV['GITHUB_KEY'], ENV['GITHUB_SECRET']
  provider:openid, :store => OpenID::Store::Filesystem.new('/tmp')
end
\end{verbatim}
By default, OmniAuth will return auth information to the path
\verb|/auth/:provider/callback| inside the Rack environment. 

In Sinatra, for
example, a callback might look something like this:

\begin{verbatim}
# Support both GET and POST for callbacks
%w(get post).each do |method|
  send(method, "/auth/:provider/callback") do
    env['omniauth.auth'] # => OmniAuth::AuthHash
  end
end
\end{verbatim}

Also of note, by default, if user authentication fails on the provider
side, OmniAuth will catch the response and then redirect the request
to the path \verb|/auth/failure|,
 passing a corresponding error message in a
parameter named \verb|message|. 

You may want to add an action to catch these
cases. Continuing with the previous Sinatra example, you could add an
action like this:

\begin{verbatim}
get '/auth/failure' do
  flash[:notice] = params[:message] # if using sinatra-flash or rack-flash
  redirect '/'
end
\end{verbatim}

\parrafo{Strategies}

In this link we can find a list of the strategies that are available for OmniAuth:
\htmladdnormallink{List of Strategies for Omniauth}{https://github.com/intridea/omniauth/wiki/List-of-Strategies}.


\subsection{Auth Hash Schema}

\htmladdnormallink{OmniAuth is a flexible authentication system utilizing Rack middleware.}{https://github.com/intridea/omniauth/wiki/Auth-Hash-Schema}

OmniAuth will always return a hash of information after authenticating
with an external provider in the Rack environment under the key
\verb|omniauth.auth|. 

This information is meant to be as normalized as
possible, so the schema below will be filled to the greatest degree
available given the provider upon authentication. Fields marked required
will always be present. 

\begin{itemize}
\item
\item \verb|provider (required)| The provider with which the user
authenticated (e.g. \verb'twitter' or \verb'facebook')

\item \verb|uid (required)| An identifier unique to the given provider,
such as a \red{Twitter user ID}. Should be stored as a string.

\item \verb|info (required)| A hash containing information about the user
  \begin{itemize}
    \item \verb|name (required)| The best display name known to the
strategy. Usually a concatenation of first and last name, but may also
be an arbitrary designator or nickname for some strategies
    \item \verb|email| The e-mail of the authenticating user. Should be provided if at all possible (but some sites such as Twitter do not provide this information)
    \item \verb|nickname| The username of an authenticating user (such
    as your \verb|@-name| from Twitter or GitHub account name)
    \item \verb|first_name|
    \item \verb|last_name|
    \item \verb|location| The general location of the user, usually a city and state.
    \item \verb|description| A short description of the authenticating user.
    \item \verb|image| A URL representing a profile image of the
    authenticating user. Where possible, should be specified to a square,
    roughly 50x50 pixel image.
    \item \verb|phone| The telephone number of the authenticating user (no formatting is enforced).
    \item \verb|urls| A hash containing key value pairs of an identifier
    for the website and its URL. 

    For instance, an entry could be 
\begin{verbatim}
"Blog" => "http://intridea.com/blog"
\end{verbatim}
  \end{itemize}
\item \verb|credentials| If the authenticating service provides some
kind of access token or other credentials upon authentication, these
are passed through here.
\item \verb|token| Supplied by OAuth and OAuth 2.0 providers, the access token.
\item \verb|secret| Supplied by OAuth providers, the access token secret.
\item \verb|extra| Contains extra information returned from the
authentication provider. May be in provider-specific formats.
\item \verb|raw_info| A hash of all information gathered about a user
in the format it was gathered. 

For example, for Twitter users this is
a hash representing the JSON hash returned from the Twitter API.
\end{itemize}

\parrafo{Ejemplos}

\begin{itemize}
\item
\htmladdnormallink{Omniauth Sinatra Example (Twitter y OpenID)}{https://github.com/intridea/omniauth/wiki/Sinatra-Example}
\item
\htmladdnormallink{Omniauth Sinatra Gist Example: Facebook, Twitter, GitHub}{https://gist.github.com/crguezl/9857758}
\end{itemize}

\parrafo{Documentación de la Gema}

\begin{itemize}
\item
\htmladdnormallink{API doc: OmniAuth: Standardized Multi-Provider Authentication}{http://rubydoc.info/github/intridea/omniauth/master/frames}
\item
\htmladdnormallink{Module: OmniAuth}{http://rubydoc.info/github/intridea/omniauth/master/OmniAuth}
\item
Esta clase contiene información sobre el usuario.
\htmladdnormallink{Class: OmniAuth::AuthHash::InfoHash}{http://rubydoc.info/github/intridea/omniauth/master/OmniAuth/AuthHash/InfoHash}
\end{itemize}

\parrafo{Véase}
\begin{itemize}
\item
\htmladdnormallink{Blog Post: ColdFusion and OAuth part 3- Google authentication.}{https://plus.google.com/+RaymondCamden/posts/idUKfLscAuV}
Raymond Camden
\item
\htmladdnormallink{}{}
\end{itemize}

\section{OmniAuth OAuth2 gem}

\begin{itemize}
\item
\htmladdnormallink{omniauth-oauth2}{https://github.com/intridea/omniauth-oauth2}
\end{itemize}

This gem contains a \blue{generic OAuth2 strategy for OmniAuth}. 

It is meant to
serve as a building block strategy for other strategies and not to be used
independently, since it has no inherent way to gather uid and user info.

\section{The gem omniauth-google-oauth2}

\parrafo{Introducción}
The 

\htmladdnormallink{gem omniauth-google-oauth2}{https://github.com/zquestz/omniauth-google-oauth2}
provides a
strategy to authenticate with Google via OAuth2 in OmniAuth.

Get your API key at: 

\htmladdnormallink{https://code.google.com/apis/console/}{https://code.google.com/apis/console/}

Note the Client ID and the Client Secret.

For more details, read the Google docs: 

\htmladdnormallink{https://developers.google.com/accounts/docs/OAuth2}{https://developers.google.com/accounts/docs/OAuth2}.

\parrafo{Configuration}

You can configure several options, which you pass in to the \verb|provider|
method via a hash:

\begin{itemize}
\item
\item \verb|scope| A \blue{comma-separated list of permissions you want to
request from the user}. 
See the 
\htmladdnormallink{Google OAuth 2.0 Playground}{https://developers.google.com/oauthplayground/}
 for a full
list of available permissions. 

Caveats:
\begin{itemize}
\item
The \verb|userinfo.email| and \verb|userinfo.profile| 
scopes are used by default. By
defining your own scope, you override these defaults. If you need these
scopes, don't forget to add them yourself!
\item
Scopes starting with \verb|https://www.googleapis.com/auth/| 
do not need that
prefix specified. 

So while you can use the smaller scope \verb|books| since that
permission starts with the mentioned prefix, you should use the full scope
URL \verb|https://docs.google.com/feeds/| 
to access a user's docs, for example.
\end{itemize}
\item \verb|prompt| A space-delimited list of string values that
determines whether the user is re-prompted for authentication and/or
consent. Possible values are:

\begin{itemize}
\item \verb|none| No authentication or consent pages will be displayed;
\red{it will return an error if the user is not already authenticated and
has not pre-configured consent for the requested scopes}. 

This can be
used as a method to check for existing authentication and/or consent.
\item \verb|consent| The user will always be prompted for consent,
even if he has previously allowed access a given set of scopes.
\item \verb|select_account| The user will always be prompted to select
a user account. This allows a user who has multiple current account
sessions to select one amongst them.
\end{itemize}

\begin{itemize}
\item
If no value is specified, the user only sees the \red{authentication page}
if he is not logged in 
\item
and only sees the \red{consent page} the first time he
authorizes a given set of scopes.
\end{itemize}

\item \verb|image_aspect_ratio| The shape of the user's profile
picture. Possible values are:

\begin{itemize}
\item \verb|original| Picture maintains its original aspect ratio.
\item \verb|square| Picture presents equal width and height.
Defaults to \verb|original|.
\end{itemize}

\item \verb|image_size| The size of the user's profile picture. 

The image
returned will have width equal to the given value and variable height,
according to the \verb|image_aspect_ratio chosen|. 

Additionally, a picture with
specific width and height can be requested by setting this option to a
hash with \verb|width| and \verb|height| as keys. 

If only width or height is specified,
a picture whose width or height is closest to the requested size and
requested aspect ratio will be returned. 

Defaults to the original width
and height of the picture.

\item \verb|name| The name of the strategy. 
The default name is
\verb|google_oauth2| but it can be changed to any value, for example \verb|google|. 
The
OmniAuth URL will thus change to \verb|/auth/google| 
and the \verb|provider| key in
the \blue{auth hash} will then return \verb|google|.

\item \verb|access_type| Defaults to \verb|offline|, 
so a refresh token is sent
to be used when the user is not present at the browser. Can be set to
\verb|online|. 

Note that if you need a refresh token, google requires you to
also to specify the option prompt: 'consent', which is not a default.

\item \verb|login_hint| When your app knows which user it is trying
to authenticate, it can provide this parameter as a hint to the
authentication server. Passing this hint suppresses the account chooser
and either pre-fill the email box on the sign-in form, or select the
proper session (if the user is using multiple sign-in), which can
help you avoid problems that occur if your app logs in the wrong user
account. The value can be either an email address or the sub string,
which is equivalent to the user's Google+ ID.

\item \verb|include_granted_scopes| 
If this is provided with the value
\verb|true|, and the authorization request is granted, the authorization will
include any previous authorizations granted to this user/application
combination for other scopes. See Google's 
\htmladdnormallink{Incremental Autorization}{https://developers.google.com/accounts/docs/OAuth2WebServer\#incrementalAuth}
for
additional details.
\end{itemize}

Here's an example of a possible configuration where 

\begin{itemize}
\item
the strategy name is changed, 
\item
the user is asked for extra permissions, 
\item
the user is always
prompted to select his account when logging in and 
\item
the user's profile
picture is returned as a thumbnail:
\end{itemize}

\begin{verbatim}
Rails.application.config.middleware.use OmniAuth::Builder do
  provider :google_oauth2, ENV["GOOGLE_CLIENT_ID"], ENV["GOOGLE_CLIENT_SECRET"],
    {
      :name => "google",
      :scope => "userinfo.email, userinfo.profile, plus.me, http://gdata.youtube.com",
      :prompt => "select_account",
      :image_aspect_ratio => "square",
      :image_size => 50
    }
end
\end{verbatim}


\parrafo{Auth Hash}
Here's an example of an authentication hash available in the callback
by accessing \verb|request.env["omniauth.auth"]|:

\begin{verbatim}
{
    :provider => "google_oauth2",
    :uid => "123456789",
    :info => {
        :name => "John Doe",
        :email => "john@company_name.com",
        :first_name => "John",
        :last_name => "Doe",
        :image => "https://lh3.googleusercontent.com/url/photo.jpg"
    },
    :credentials => {
        :token => "token",
        :refresh_token => "another_token",
        :expires_at => 1354920555,
        :expires => true
    },
    :extra => {
        :raw_info => {
            :id => "123456789",
            :email => "user@domain.example.com",
            :verified_email => true,
            :name => "John Doe",
            :given_name => "John",
            :family_name => "Doe",
            :link => "https://plus.google.com/123456789",
            :picture => "https://lh3.googleusercontent.com/url/photo.jpg",
            :gender => "male",
            :birthday => "0000-06-25",
            :locale => "en",
            :hd => "company_name.com"
        }
    }
}
\end{verbatim}


\section{Using OAuth 2.0 to Access Google APIs}

Véase 
\htmladdnormallink{Using OAuth 2.0 to Access Google APIs}{https://developers.google.com/accounts/docs/OAuth2}.

Google APIs use the \OAuth{} 2.0 protocol for authentication and
authorization. Google supports common \OAuth{} 2.0 scenarios such as those
for web server, installed, and client-side applications.

\begin{enumerate}
\item
\OAuth{} 2.0 is a relatively simple protocol. To begin, you obtain \OAuth{}
2.0 credentials from the 
\htmladdnormallink{Google Developers Console}{https://cloud.google.com/console}. 
\item
See the Video in YouTube 
\htmladdnormallink{Obtaining a developer key for the YouTube Data API v3 and the Analytics API}{http://youtu.be/Im69kzhpR3I}
\item
Then your client
application requests an \blue{access token} from the 
\blue{Google Authorization Server},
extracts a token from the response, and sends the token to the 
\blue{Google API} 
that you want to access.
\end{enumerate}

To get access keys, go to the Google APIs Console and specify your
\htmladdnormallink{Google Developers Console}{https://cloud.google.com/console}. 
application's name and the Google APIs it will access. For simple access,
Google generates an API key that uniquely identifies your application
in its transactions with the Google Auth server.

For authorized access, you must also tell Google your website's protocol
and domain. In return, Google generates a client ID. Your application
submits this to the Google Auth server to get an \OAuth{} 2.0 access
token.

\section{Google OAuth 2.0 Playground}
\htmladdnormallink{Google OAuth 2.0 Playground}{https://developers.google.com/oauthplayground/}

\section{Sign-in with Google +} 
Google+ Sign-in provides the \OAuth{} 2.0 authentication mechanism along
with additional access to Google desktop and mobile features.
\begin{itemize}
\item
\htmladdnormallink{https://developers.google.com/+/}{https://developers.google.com/+/}
provides the \OAuth{} 2.0 authentication mechanism along with additional access to Google desktop and mobile features.
\item
Direct access to an authentication service based on the standardized OpenID Connect mechanism is also available.
\htmladdnormallink{https://developers.google.com/accounts/docs/OAuth2Login}{https://developers.google.com/accounts/docs/OAuth2Login}
\end{itemize}

\section{Revoking Access to an App}

Go to
\htmladdnormallink{https://security.google.com/settings/security/permissions}{https://security.google.com/settings/security/permissions?pli=1}

Or to your configuration 
\htmladdnormallink{https://www.google.com/settings/personalinfo}{https://www.google.com/settings/personalinfo}
and there to {\bf security}. From there go to the section
{\bf Account permissions} ({\it Control which apps and websites have access
to your account information}). Choose the link {\bf View All}.

\begin{itemize}
\item
Remember to kill the session if you want to check that the permit
has been effectively removed. 
\item
Go to chrome and open a window in \red{incognito mode}
\item
Attempt to access the URL that requires Google Authentication
in your service
\item
It has to ask you for permits again
\end{itemize}





\section{Google + API for Ruby}

\parrafo{Donde}

\begin{itemize}
\item
\htmladdnormallink{https://github.com/crguezl/gplus-quickstart-ruby}{https://github.com/crguezl/gplus-quickstart-ruby}
\item
\begin{verbatim}
[~/sinatra/gplus-quickstart-ruby(master)]$ pwd -P
/Users/casiano/local/src/ruby/sinatra/gplus-quickstart-ruby
[~/sinatra/gplus-quickstart-ruby(master)]$ git remote -v
googleplus      git@github.com:googleplus/gplus-quickstart-ruby.git (fetch)
googleplus      git@github.com:googleplus/gplus-quickstart-ruby.git (push)
origin  git@github.com:crguezl/gplus-quickstart-ruby.git (fetch)
origin  git@github.com:crguezl/gplus-quickstart-ruby.git (push)
\end{verbatim}
\item
\htmladdnormallink{https://github.com/crguezl/gplus-quickstart-ruby}{https://github.com/crguezl/gplus-quickstart-ruby}
\item
\htmladdnormallink{https://developers.google.com/+/quickstart/ruby}{https://developers.google.com/+/quickstart/ruby}
\end{itemize}

The app demonstrates:

\begin{itemize}
\item
Using the Google+ Sign-In button to get an \OAuth{} 2.0 refresh token.
\item
Exchanging the refresh token for an access token.
\item
Making Google+ API requests with the access token, including getting a list of people that the user has circled.
\item
Disconnecting the app from the user's Google account and revoking tokens.
\end{itemize}

\parrafo{Step 1: Enable the Google+ API}

Create a Google APIs Console project, \OAuth{} 2.0 client ID, and register your JavaScript origins:

To register a new application, do the following:

\begin{itemize}
\item
Go to the Google Cloud Console.
\item
Select a project, or create a new one.
\item
In the sidebar on the left, select \verb|APIs & auth|. 
In the displayed list of APIs, make sure the \verb|Google+ API| status is set to \verb|ON|.
\item
In the sidebar on the left, select \verb|Registered apps|.
\item
At the top of the page, select Register App.
\item
Fill out the form and select Register.
\end{itemize}
\red{Register the origins where your app is allowed to access the Google APIs}. 
The origin is the unique combination of protocol, hostname, and
port. You can enter multiple origins to allow for your app to run on
different protocols, domains or subdomains. Wildcards are not allowed.

\begin{itemize}
\item
Expand the \OAuth{} 2.0 Client ID section.
\item
In the Web origin field, enter your origin:
\begin{verbatim}
http://localhost:4567
\end{verbatim}
\item
Press ENTER to save your origin. You can then click the + symbol to add additional origins.
\item
Note or copy the client ID and client secret that your app will need to use to access the APIs.
\end{itemize}

\htmladdnormallink{Client Secrets}{https://developers.google.com/api-client-library/ruby/guide/aaa_client_secrets}

\section{Google+ Sign-In for server-side apps}

To take advantage of all of the benefits of Google+ Sign-In you must
use a hybrid server-side flow where a user authorizes your app on the
client side using the JavaScript API client and you send a
special one-time authorization code to your server. 

Your server exchanges this one-time-use code to acquire its own access and
refresh tokens from Google for the server to be able to make its own API
calls, which can be done while the user is offline. 

This one-time code flow has security advantages over both a pure
server-side flow and over sending access tokens to your server.

\red{The Google+ Sign-In server-side flow differs from the \OAuth{} 2.0
for Web server applications flow.}

\begin{rawhtml}
<img src="server_side_code_flow.png" />
\end{rawhtml}

\section{Authentication using the Google APIs Client Library for JavaScript}

\htmladdnormallink{See}{https://developers.google.com/api-client-library/javascript/features/authentication}.

