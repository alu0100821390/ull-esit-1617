\parrafo{Local Storage (HTML5 Web Storage)}

\cei{Web storage} and \cei{DOM storage} 
(document object model) are web application
software methods and protocols used for storing data in a web browser. 

\begin{itemize}
\item
Web storage supports persistent data storage, similar to cookies but with
a greatly enhanced capacity and no information stored in the HTTP
request header.

\item
Local Storage nos permite almacenar hasta 5MB del lado del cliente
por dominio, esto nos permite ahora hacer aplicaciones mas robustas y
con mas posibilidades. Las Cookies ofrecen algo
parecido, pero con el limite de 100kb.

\item
There are two main web storage types: \cei{local storage}
and \cei{session storage}, behaving similarly to persistent cookies and session
cookies respectively.
\item
Unlike cookies, which can be accessed by both the server and client side, web storage falls exclusively under the purview of client-side scripting
\item
The HTML5 localStorage object is isolated per domain (the same segregation rules as the \wikip{same origin policy}{Same\_origin\_policy}).
Under this policy, \blue{a web browser permits scripts contained in a first web page to access data in a second web page, but only if both web pages have the same origin}. 

The same-origin policy \red{permits scripts running on pages originating from the same
site} – \blue{a combination of scheme, hostname, and port number} – \red{to access
each other's DOM with no specific restrictions}, {\bf but prevents access to
DOM on different sites}.
\end{itemize}

Véase:
\begin{itemize}
\item
Ejemplo en  GitHub:
\htmladdnormallink{https://github.com/crguezl/web-storage-example}{https://github.com/crguezl/web-storage-example}
\begin{verbatim}
[~/javascript/local_storage(master)]$ pwd -P
/Users/casiano/local/src/javascript/local_storage
\end{verbatim}
\item 
\htmladdnormallink{Como usar localstorage}{http://html5facil.com/tutoriales/como-usar-local-storage-de-javascript}

\item 
\htmladdnormallink{HTML5 Web Storage}{http://www.w3schools.com/html/html5_webstorage.asp}
\item
\htmladdnormallink{W3C Web Storage}{http://www.w3.org/TR/webstorage/}
\item
\htmladdnormallink{Using HTML5 localStorage To Store JSON}{http://getfishtank.ca/blog/using-html5-localstorage-to-store-json}
Options for persistent storage of complex JavaScript objects in HTML5
by Dan Cruickshank
\item
\htmladdnormallink{HTML5 Cookbook.  Christopher Schmitt, Kyle Simpson "O'Reilly Media, Inc.", Nov 7, 2011}{http://books.google.es/books/about/HTML5_Cookbook.html?id=cXcaY7XVZbcC&redir_esc=y}
Chapter 10. Section 2: LocalStorage
\end{itemize}

While Chrome does not provide a UI for clearing localStorage, there is an API that will either clear a specific key or the entire localStorage object on a website.

\begin{verbatim}
//Clears the value of MyKey
window.localStorage.clear("MyKey");

//Clears all the local storage data
window.localStorage.clear();
\end{verbatim}
Once done, localStorage will be cleared. Note that this affects
all web pages on a single domain, so if you clear localStorage for
\verb|jsfiddle.net/index.html| (assuming that's the page you're on), then it
clears it for all other pages on that site. 


