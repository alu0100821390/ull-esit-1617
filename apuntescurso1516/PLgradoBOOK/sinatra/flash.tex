\htmladdnormallink{Sinatra::Flash}{https://github.com/SFEley/sinatra-flash}
is an extension that lets you store information between
requests. 

Often, when an application processes a request, it will redirect
to another URL upon finishing, which generates another request. 

This
means that any information from the previous request is lost (due to the
stateless nature of HTTP). 

\htmladdnormallink{Sinatra::Flash}{https://github.com/SFEley/sinatra-flash}
overcomes this by providing
access to the \red{flash}—a hash-like object that stores temporary values
such as error messages so that they can be retrieved later—usually
on the next request. 

It also removes the information once it’s been
used. 

All this can be achieved via sessions (and that’s exactly how
\htmladdnormallink{Sinatra::Flash}{https://github.com/SFEley/sinatra-flash}
does it), but 
\htmladdnormallink{Sinatra::Flash}{https://github.com/SFEley/sinatra-flash}
is easy to implement and
provides a number of helper methods.

\parrafo{Ejemplo}
\begin{verbatim}
[~/sinatra/sinatra-flash]$ cat app.rb 
require 'sinatra'
require 'sinatra/flash'

enable :sessions

get '/blah' do
  # This message won't be seen until the NEXT Web request that accesses the flash collection
  flash[:blah] = "You were feeling blah at #{Time.now}."

  # Accessing the flash displays messages set from the LAST request
  "Feeling blah again? That's too bad. #{flash[:blah]}"
end

get '/pum' do
  # This message won't be seen until the NEXT Web request that accesses the flash collection
  flash[:pum] = "You were feeling pum at #{Time.now}."

  # Accessing the flash displays messages set from the LAST request
  "Feeling pum again? That's too bad. #{flash[:pum]}"
end
\end{verbatim}

\parrafo{Gemfile}
\begin{verbatim}
[~/sinatra/sinatra-flash]$ cat Gemfile
source 'https://rubygems.org'

gem 'sinatra'
gem 'sinatra-flash'
\end{verbatim}
