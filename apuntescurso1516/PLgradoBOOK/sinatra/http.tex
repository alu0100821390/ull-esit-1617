\subsection{Introducción}

\begin{enumerate}
\item 
HTTP es un protocolo sin estado: que no guarda ninguna
información sobre conexiones anteriores. 

\item 
El desarrollo de aplicaciones
web necesita frecuentemente mantener estado. 

\item 
Para esto se usan las
cookies, que es información que un servidor puede almacenar en el
sistema cliente. 

\item 
Esto le permite a las aplicaciones web introducir
la noción de \cei{sesión}, y también permite rastrear usuarios ya que
las cookies pueden guardarse en el cliente por tiempo indeterminado.
\item
Una transacción HTTP está formada por un \cei{encabezado} seguido,
opcionalmente, por una línea en blanco y algún dato. 

\item 
El encabezado
especificará cosas como la acción requerida del servidor, o el tipo
de dato retornado, o el código de estado.
\item 
El uso de campos de encabezados enviados en las transacciones HTTP
le da flexibilidad al protocolo. Estos campos permiten que
se envíe información descriptiva en la transacción, permitiendo así
la autenticación, cifrado e identificación de usuario.
\item 
Un encabezado es un bloque de datos que precede a la información
propiamente dicha, por lo que a veces se hace referencia a él
como metadato, porque tiene datos sobre los datos.
\item 
Si se reciben líneas de encabezado del cliente, el servidor las
coloca en las variables de entorno de CGI con el prefijo \verb|HTTP_|
seguido del nombre del encabezado. Cualquier carácter guion ( \verb|-| )
del nombre del encabezado se convierte a caracteres \verb|"_"|.
%\item 
%El servidor puede excluir cualquier encabezado que ya esté procesado,
%como 
%\verb|Authorization|, 
%\verb|Content-type| y 
%\verb|Content-length|. 
%\item 
%El servidor puede
%elegir excluir alguno o todos los encabezados.

Ejemplos de estos encabezados del cliente son
\verb|HTTP_ACCEPT| y 
\verb|HTTP_USER_AGENT|.
  \begin{enumerate}
  \item 
  \verb|HTTP_ACCEPT|. Los tipos MIME que el cliente aceptará, dados
  los encabezados HTTP. 
  Los elementos de esta lista deben
  estar separados por comas
  \item 
  \verb|HTTP_USER_AGENT|. El navegador que utiliza el cliente para
  realizar la petición. El formato general para esta variable es:
  software/versión biblioteca/versión.
  \end{enumerate}
El servidor envía al cliente:
  \begin{enumerate}
  \item 
  Un \cei{código de estado} que indica si la petición fue correcta o no.
  Los códigos de error típicos indican que el archivo solicitado no
  se encontró, que la petición no se realizó de forma correcta o que
  se requiere autenticación para acceder al archivo.
  \item 
  La información propiamente dicha. HTTP permite enviar documentos
  de todo tipo y formato, como
  gráficos, audio y video.
  \item 
  Información sobre el objeto que se retorna.
  \end{enumerate}
\end{enumerate}

\subsection{Sesiones HTTP}
\begin{enumerate}
\item 
Una sesión HTTP es una secuencia de transacciones de red de peticiones y respuestas

\item 
Un cliente HTTP inicia una petición estableciendo una conexión TCP con un puerto 
particular de un servidor (normalmente el puerto 80)
\item 
Un servidor que esté escuchando en ese puerto espera por un mensaje
de petición de un cliente.
\item 
El servidor retorna la \cei{línea de estatus}, por ejemplo \verb"HTTP/1.1 200 OK",
y su propio mensaje. El cuerpo de este mensaje suele ser el recurso
solicitado, aunque puede que se trate de un mensaje de error u otro tipo de información.
\end{enumerate}

Veamos un ejemplo. Usemos este servidor:
\begin{verbatim}
[~/local/src/ruby/sinatra/rack/rack-debugging]$ cat hello1.rb 
require 'rack'

class HelloWorld
  def call env
    [200, {"Content-Type" => "text/plain"}, ["Hello world"]]
  end
end

Rack::Handler::WEBrick::run HelloWorld.new
\end{verbatim}


\begin{verbatim}
[~/local/src/ruby/sinatra/rack/rack-debugging]$ ruby hello1.rb 
[2013-09-23 15:16:58] INFO  WEBrick 1.3.1
[2013-09-23 15:16:58] INFO  ruby 1.9.3 (2013-02-22) [x86_64-darwin11.4.2]
[2013-09-23 15:16:58] INFO  WEBrick::HTTPServer#start: pid=12113 port=8080

\end{verbatim}

Arrancamos un cliente con telnet con la salida redirigida:
\begin{verbatim}
[~/local/src/ruby/sinatra/rack/rack-debugging]$ telnet localhost 8080 > salida
\end{verbatim}

Escribimos esto en la entrada estandard:
\begin{verbatim}
GET /index.html HTTP/1.1
Host: localhost
Connection: close


\end{verbatim}
con una línea en blanco al final.
Este texto es enviado al servidor.

El cliente deja su salida en el fichero \verb|salida|:
\begin{verbatim}
[~/local/src/ruby/sinatra/rack/rack-debugging]$ cat salida
Trying ::1...
Connected to localhost.
Escape character is '^]'.
HTTP/1.1 200 OK 
Content-Type: text/plain
Server: WEBrick/1.3.1 (Ruby/1.9.3/2013-02-22)
Date: Mon, 23 Sep 2013 14:33:16 GMT
Content-Length: 11
Connection: close

Hello world
\end{verbatim}
El cliente escribe en la salida estandard:
\begin{verbatim}
Connection closed by foreign host.
\end{verbatim}

\subsection{Métodos de Petición}
\label{subsection:metodosdepeticionhttp}
\begin{enumerate}
\item \cei{GET}

Solicita una representación de un recurso especificado.
Las peticiones que usen \cei{GET} deberían limitarse a obtener los datos 
y no tener ningún otro efecto.
\item \cei{HEAD}

Pregunta por la misma respuesta que una petición \cei{GET} pero sin el 
cuerpo de la respuesta
\item \cei{POST}

Requests that the server accept the entity enclosed in the request
as a new subordinate of the web resource identified by the URI. The
data POSTed might be, as examples, 
  \begin{enumerate}
  \item 
  an annotation for existing
  resources; 
  \item 
  a message for a bulletin board, newsgroup, mailing list,
  or comment thread; 
  \item 
  a block of data that is the result of submitting
  a web form to a data-handling process; 
  \item 
  or an item to add to a
  database.
  \end{enumerate}
\item \cei{PUT}

Requests that the enclosed entity be stored under the supplied URI.
If the URI refers to an already existing resource, it is modified;
if the URI does not point to an existing resource, then the server
can create the resource with that URI.
\item \cei{DELETE}

Deletes the specified resource.
\item \cei{TRACE}

Echoes back the received request so that a client can see what (if
any) changes or additions have been made by intermediate servers.
\item \cei{OPTIONS}

Returns the HTTP methods that the server supports for the specified
URL. This can be used to check the functionality of a web server
by requesting \verb'*' instead of a specific resource.
\item \cei{CONNECT}

Converts the request connection to a transparent TCP/IP tunnel,
usually to facilitate SSL-encrypted communication (HTTPS) through
an unencrypted HTTP proxy.
\item \cei{PATCH}

Is used to apply partial modifications to a resource.
HTTP servers are required to implement at least the GET and HEAD
methods and, whenever possible, also the OPTIONS method
\end{enumerate}

\subsection{Véase}

\begin{enumerate}
\item 
\htmladdnormallink{ArrrrCamp \#6 - Konstantin Haase - We don't know HTTP}{https://vimeo.com/51903877}
\item 
\htmladdnormallink{Resources, For Real This Time (with Webmachine) Sean Cribbs}{http://www.confreaks.com/videos/699-rubyconf2011-resources-for-real-this-time-with-webmachine} Ruby Conference 2011
\end{enumerate}
