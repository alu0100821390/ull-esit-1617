
\section{Introducción}
CoffeeScript is a programming language that looks like this: 

\begin{verbatim}
[~/coffee/jump_start_coffeescript/chapter01/casiano(master)]$ coffee
coffee> hello = (name) -> "Hello, #{name}!"
[Function]
coffee> hello('world!')
'Hello, world!!'
coffee> hello 'world!'
'Hello, world!!'
\end{verbatim}

Here,
we’re defining and then calling a function, \verb|hello|, which accepts a
single parameter, \verb|name|.

\parrafo{Instalación}

The CoffeeScript compiler is itself written in CoffeeScript, using
the Jison parser generator. The command-line version of coffee is
available as a Node.js utility. The core compiler however, does not
depend on Node, and can be run in any JavaScript environment, or
in the browse.

To install, first make sure you have a working copy of the latest
stable version of Node.js, and npm (the Node Package Manager). You
can then install CoffeeScript with npm:

\begin{verbatim}
npm install -g coffee-script
\end{verbatim}
(Leave off the -g if you don't wish to install globally.)

If you'd prefer to install the latest master version of CoffeeScript,
you can clone the CoffeeScript source repository from GitHub, or
download the source directly. To install the lastest master
CoffeeScript compiler with npm:

\begin{verbatim}
npm install -g http://github.com/jashkenas/coffee-script/tarball/master
\end{verbatim}
Or, if you want to install to /usr/local, and don't want to use npm
to manage it, open the coffee-script directory and run:

\begin{verbatim}
sudo bin/cake install
\end{verbatim}

CoffeeScript includes a (very) simple build system similar to Make
and Rake. Naturally, it's called \tei{Cake}, and is used for the tasks
that build and test the CoffeeScript language itself. Tasks are
defined in a file named Cakefile, and can be invoked by running
\verb|cake [task]| from within the directory. To print a list of all the
tasks and options, just type \verb|cake|.

\parrafo{Enlaces Relacionados}
\begin{enumerate}
\item 
\htmladdnormallink{CoffeeScript book}{http://coffeescriptcookbook.com/}
\item 
\htmladdnormallink{Railcast: CoffeeScript}{http://railscasts.com/episodes/267-coffeescript-basics}
\item 
\htmladdnormallink{vim plugin para CoffeScript}{https://github.com/kchmck/vim-coffee-script}
\item
\htmladdnormallink{A CoffeeScript Intervention.
Five Things You Thought You Had to Live with in JavaScript}{http://pragprog.com/magazines/2011-05/a-coffeescript-intervention}
por Trevor Burnham en 
\htmladdnormallink{PragPub}{http://pragprog.com/magazines/}
\item
\htmladdnormallink{A taste of CoffeeScript (part 2)}{http://www.webdesignerdepot.com/2013/05/a-taste-of-coffeescript-part-2/}
\item
\htmladdnormallink{Some real world examples of Coffescript and jQuery}{http://bleibinha.us/blog/2013/06/some-real-world-examples-of-coffescript-and-jquery} por Stefan Bleibinhaus
\item
\htmladdnormallink{js2coffee: Compile JavaScript to CoffeeScript}{https://github.com/js2coffee/js2coffee}
\item
\htmladdnormallink{I can't write multiline codes in Coffeescript Interactive Mode(REPL)}{http://stackoverflow.com/questions/10491849/i-cant-write-multiline-codes-in-coffeescript-interactive-moderepl}:
ctrl-V
\item 
\htmladdnormallink{Railcast: CoffeeScript}{http://railscasts.com/episodes/267-coffeescript-basics}
\end{enumerate}


\section{CoffeeScript y JQuery}

Es posible instalar 
\htmladdnormallink{JQuery en Node.js}{https://npmjs.org/package/jquery}.
Tuve algún problema para instalar jquery con algunas versiones de Node pero funcionó
con la 0.10.10:
\begin{verbatim}
~/sinatra/rockpaperscissors(master)]$ n
  * 0.10.10 
    0.11.2 
    0.8.17
\end{verbatim}
El programa
\htmladdnormallink{n}{https://github.com/visionmedia/n}
es un gestor de versiones de Node.js. Otro gestor de versiones del intérprete es
\htmladdnormallink{nvm}{https://github.com/creationix/nvm}.

Una vez instalado, podemos usarlo desde coffeescript via node.js:
\begin{verbatim}
coffee> $ = require 'jquery'; null
null
coffee> $("<h1>test passes</h1>").appendTo "body" ; null
null
coffee> console.log $("body").html() 
<h1>test passes</h1>
undefined
coffee> 
coffee> $.each [4,3,2,1], (i,v)-> console.log "index: #{i} -> value: #{v}"
index: 0 -> value: 4
index: 1 -> value: 3
index: 2 -> value: 2
index: 3 -> value: 1
[ 4, 3, 2, 1 ]
\end{verbatim}

\section{Ambito/Scope}

\begin{itemize}
\item
\htmladdnormallink{Lexical Scope in CoffeeScript}{https://github.com/raganwald/homoiconic/blob/master/2012/09/lexical-scope-in-coffeescript.md} por 
\htmladdnormallink{Reg Braithwaite raganwald}{http://raganwald.com/}

\item
\htmladdnormallink{Reg Braithwaite raganwald}{http://braythwayt.com/}
\item
\htmladdnormallink{ristrettolo.gy, CoffeeScript Ristretto Online}{http://ristrettolo.gy/}
\end{itemize}

\section{Cake}

Cakefiles are inspired by the venerable Makefiles, the build system for
C and friends. 

The premise of a Makefile and its variants (Ruby users will also be familiar with Rakefiles) is simple: it consists of various tasks which can be invoked individually from the command line. 

Each task definition contains instructions that will be run when the task is run. In a Cakefile, the task definitions and bodies are all written in CoffeeScript.


\parrafo{See}
\begin{itemize}
\item
\htmladdnormallink{HowTo: Compiling and Setting Up Build Tools}{https://github.com/jashkenas/coffeescript/wiki/\%5BHowTo\%5D-Compiling-and-Setting-Up-Build-Tools}
in repo jashkenas/coffeescript
\end{itemize}

