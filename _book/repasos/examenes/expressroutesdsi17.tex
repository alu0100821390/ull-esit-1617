\subsection{Rutas en Express}\label{rutas-en-express}

\begin{enumerate}
\def\labelenumi{\arabic{enumi}.}
\itemsep1pt\parskip0pt\parsep0pt
\item
  Escriba un servidor que sirva ficheros estáticos desde el directorio
  \texttt{/public}
\item
  El servidor deberá responder a requests \texttt{GET} en las rutas
  \texttt{/user/nombredeusuario} (donde \texttt{nombredeusuario} varía)
  con una página que diga \texttt{Hola nombredeusuario} usando el método
  \texttt{render} del objeto \texttt{response}
\end{enumerate}

\begin{itemize}
\itemsep1pt\parskip0pt\parsep0pt
\item
  La página se elaborara con una vista que debe estar en el directorio
  \texttt{views/} usando el motor de vistas \texttt{ejs}
\item
  La página elaborada en la respuesta tendrá un tag \texttt{img}
  referenciando a una imagen que está en \texttt{public/}
\end{itemize}

\begin{enumerate}
\def\labelenumi{\arabic{enumi}.}
\setcounter{enumi}{2}
\itemsep1pt\parskip0pt\parsep0pt
\item
  Escriba un middleware que intercepte en las rutas
  \texttt{/user/nombredeusuario} y que vuelque en la consola información
  sobre el \href{https://expressjs.com/en/4x/api.html\#req}{request}:
  (por ejemplo los atributos \texttt{method}, \texttt{path}, etc.)
\item
  Explique como se puede aislar el código anterior en un fichero
  \texttt{routes/user.js} que sea cargado desde el programa principal
\item
  Explique que hay que hacer para desplegar la aplicación en la máquina
  virtual del iaas
\end{enumerate}
