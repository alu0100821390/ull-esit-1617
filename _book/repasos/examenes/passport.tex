\begin{itemize}
\itemsep1pt\parskip0pt\parsep0pt
\item
  ¿Que es OAuth?
\item
  OAuth provides a method for users to grant third-party limited access
  (in scope, duration, etc.) access to their resources without sharing
  their passwords
\item
  ¿Quienes son los cuatro roles que aparecen en una autenticación con
  OAuth?
\end{itemize}

\begin{enumerate}
\def\labelenumi{\arabic{enumi}.}
\itemsep1pt\parskip0pt\parsep0pt
\item
  resource owner: An entity capable of granting access to a protected
  resource. When the resource owner is a person, it is referred to as an
  end-user. (El usuario)
\item
  resource server: The server hosting the protected resources, capable
  of accepting and responding to protected resource requests using
  access tokens. (El servidor de Pinterest)
\item
  client: An application making protected resource requests on behalf of
  the resource owner and with its authorization (por ejemplo, un cliente
  de pinterest en el teléfono). The term " client" does not imply any
  particular implementation characteristics (e.g., whether the
  application executes on a server, a desktop, or other devices).
\item
  authorization server: The server issuing access tokens to the client
  after successfully authenticating the resource owner and obtaining
  authorization (por ejemplo, Facebook, cuando nos autenticamos con
  Facebook)
\end{enumerate}

\begin{itemize}
\itemsep1pt\parskip0pt\parsep0pt
\item
  ¿Qué tres elementos de información suelen ser necesarios a la hora de
  registrar nuestra aplicación ante un proveedor de OAuth?
\item
  Before using OAuth with your application, you must register your
  application with the service.
\item
  This is done through a registration form in the developer or API
  portion of the service's website, where you will provide the following
  information (and probably details about your application):

  \begin{enumerate}
  \def\labelenumi{\arabic{enumi}.}
  \itemsep1pt\parskip0pt\parsep0pt
  \item
    Application Name
  \item
    Application Website
  \item
    Redirect URI or Callback URL
  \end{enumerate}
\item
  ¿Que se debe poner en \emph{Redirect URI or Callback URL} cuando se
  está registrando nuestra aplicación?
\item
  The redirect URI is where the service will redirect the user after
  they authorize (or deny) your application, and therefore the part of
  your application that will handle authorization codes or access
  tokens.
\item
  Una vez que registramos la aplicación, el servicio provee las
  credenciales del cliente. ¿En que consisten esas credenciales?
\item
  Once your application is registered, the service will issue client
  credentials in the form of a client identifier and a client secret.
\item
  The Client ID is a publicly exposed string that is used by the service
  API to identify the application, and is also used to build
  authorization URLs that are presented to users.
\item
  The Client Secret is used to authenticate the identity of the
  application to the service API when the application requests to access
  a user's account, and must be kept private between the application and
  the API.
\item
  ¿Que es \emph{passport}, que funcionalidades provee y como funciona?
\item
  Passport is authentication middleware for Node.js. Extremely flexible
  and modular, Passport can be unobtrusively dropped in to any
  Express-based web application. A comprehensive set of strategies
  support authentication using a username and password, Facebook,
  Twitter, and more.
\item
  Rellene las partes que faltan:
\end{itemize}

```javascript var passport = require(`passport'); var Strategy =
require(`\_\_\_\_\_\_\_\_\_\_\_\_\_\_\_').Strategy; var github =
require(`octonode'); \ldots{}. var datos\_config =
JSON.parse(JSON.stringify(config));

passport.use(new Strategy(\{ clientID: datos\_config.clientID,
clientSecret: datos\_config.clientSecret, callbackURL: callbackURL\_ \},
function(accessToken, refreshToken, profile, cb) \{

\begin{verbatim}
    var token = datos_config.token;
    var client = github.client(_____);

    var ghorg = client.___('ULL-ESIT-SYTW-1617');

    ghorg.______(profile.username, (err,result) =>
    {
        if(err) console.log(err);
        console.log("Result:"+result);
        if(result == true)
          return cb(null, profile);
        else {
          return cb(null,null);
        }
    });
\end{verbatim}

\})); ``` - Respuesta:

```javascript var passport = require(`passport'); var Strategy =
require(`passport-github').Strategy; var github = require(`octonode');
\ldots{}. var datos\_config = JSON.parse(JSON.stringify(config));

passport.use(new Strategy(\{ clientID: datos\_config.clientID,
clientSecret: datos\_config.clientSecret, callbackURL: callbackURL\_ \},
function(accessToken, refreshToken, profile, cb) \{

\begin{verbatim}
    var token = datos_config.token;
    var client = github.client(token);

    var ghorg = client.org('ULL-ESIT-SYTW-1617');

    ghorg.member(profile.username, (err,result) =>
    {
        if(err) console.log(err);
        console.log("Result:"+result);
        if(result == true)
          return cb(null, profile);
        else {
          return cb(null,null);
        }
    });
  // return cb(null, profile);
\end{verbatim}

\})); ```
