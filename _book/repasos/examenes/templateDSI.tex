\documentclass[]{article}
\usepackage{lmodern}
\usepackage{longtable}
\usepackage{booktabs}
\usepackage{amssymb,amsmath}
\usepackage{ifxetex,ifluatex}
\usepackage{fixltx2e} % provides \textsubscript
\ifnum 0\ifxetex 1\fi\ifluatex 1\fi=0 % if pdftex
  \usepackage[T1]{fontenc}
  \usepackage[utf8]{inputenc}
\else % if luatex or xelatex
  \ifxetex
    \usepackage{mathspec}
    \usepackage{xltxtra,xunicode}
  \else
    \usepackage{fontspec}
  \fi
  \defaultfontfeatures{Mapping=tex-text,Scale=MatchLowercase}
  \newcommand{\euro}{€}
\fi
% use upquote if available, for straight quotes in verbatim environments
\IfFileExists{upquote.sty}{\usepackage{upquote}}{}
% use microtype if available
\IfFileExists{microtype.sty}{%
\usepackage{microtype}
\UseMicrotypeSet[protrusion]{basicmath} % disable protrusion for tt fonts
}{}
\usepackage{color}
\usepackage{fancyvrb}
\newcommand{\VerbBar}{|}
\newcommand{\VERB}{\Verb[commandchars=\\\{\}]}
\DefineVerbatimEnvironment{Highlighting}{Verbatim}{commandchars=\\\{\}}
% Add ',fontsize=\small' for more characters per line
\newenvironment{Shaded}{}{}
\newcommand{\KeywordTok}[1]{\textcolor[rgb]{0.00,0.44,0.13}{\textbf{{#1}}}}
\newcommand{\DataTypeTok}[1]{\textcolor[rgb]{0.56,0.13,0.00}{{#1}}}
\newcommand{\DecValTok}[1]{\textcolor[rgb]{0.25,0.63,0.44}{{#1}}}
\newcommand{\BaseNTok}[1]{\textcolor[rgb]{0.25,0.63,0.44}{{#1}}}
\newcommand{\FloatTok}[1]{\textcolor[rgb]{0.25,0.63,0.44}{{#1}}}
\newcommand{\CharTok}[1]{\textcolor[rgb]{0.25,0.44,0.63}{{#1}}}
\newcommand{\StringTok}[1]{\textcolor[rgb]{0.25,0.44,0.63}{{#1}}}
\newcommand{\CommentTok}[1]{\textcolor[rgb]{0.38,0.63,0.69}{\textit{{#1}}}}
\newcommand{\OtherTok}[1]{\textcolor[rgb]{0.00,0.44,0.13}{{#1}}}
\newcommand{\AlertTok}[1]{\textcolor[rgb]{1.00,0.00,0.00}{\textbf{{#1}}}}
\newcommand{\FunctionTok}[1]{\textcolor[rgb]{0.02,0.16,0.49}{{#1}}}
\newcommand{\RegionMarkerTok}[1]{{#1}}
\newcommand{\ErrorTok}[1]{\textcolor[rgb]{1.00,0.00,0.00}{\textbf{{#1}}}}
\newcommand{\NormalTok}[1]{{#1}}
\ifxetex
  \usepackage[setpagesize=false, % page size defined by xetex
              unicode=false, % unicode breaks when used with xetex
              xetex]{hyperref}
\else
  \usepackage[unicode=true]{hyperref}
\fi
\hypersetup{breaklinks=true,
            bookmarks=true,
            pdfauthor={},
            pdftitle={},
            colorlinks=true,
            citecolor=blue,
            urlcolor=blue,
            linkcolor=magenta,
            pdfborder={0 0 0}}
\urlstyle{same}  % don't use monospace font for urls
\setlength{\parindent}{0pt}
\setlength{\parskip}{6pt plus 2pt minus 1pt}
\setlength{\emergencystretch}{3em}  % prevent overfull lines
\setcounter{secnumdepth}{0}

\usepackage{lastpage}

\date{}

\begin{document}

\thispagestyle{empty}
%begin{tabular}{lcc}
%%%%
% \begin{tabular}{c}
%   \epsfig{file=/tmp/ullesc.eps,width=1.5cm}  
% \end{tabular}                      &
%%%%
  \begin{tabular}{c}
   {\bf Universidad de La Laguna.  Escuela Técnica Superior de Ingeniería Informática}     \\
   {\bf Tercero del Grado de Informática}\\
   {\bf DESARROLLO DE SISTEMAS INFORMÁTICOS: 1ª PARTE}\\
   07/03/2017  \pageref*{LastPage} páginas         \\   
  \end{tabular}                     % &
%%%%
%%%%
%end{tabular}

\bigskip

%\hrulefill
Nombre:  \underline{\hspace{11.5cm}} Fecha 07/03/2017\underline{\hspace{2cm}} 
\bigskip

%\begin{footnotesize}
%%Notas:
%\begin{itemize}
%  \item
%%  La duración del examen completo es de 2 horas.
%   Respete el uso de mayúsuclas y minúsculas en los comandos y programas
%  \item Escriba con letra clara. Use también el reverso de las hojas 
%  \item Los ejercicios deben realizarse con bolígrafo.
%  \item Al finalizar el exámen, ENTREGAR TODOS LOS FOLIOS utilizados, incluyendo éste.
%  \item Las calificaciones del exámen estarán disponibles unos días antes de la fecha límite de entrega de las actas.
%%  \item Si esta es su 5ª ó 6ª convocatoria, escriba “Xª CONVOCATORIA” en el encabezado de esta hoja.
%\end{itemize}
%\end{footnotesize}



%\tableofcontents

\section{MarkDown}
\subsection{Repaso de MarkDown}\label{repaso-de-markdown}

\begin{itemize}
\item
  ¿Como se escribe en MarkDown un link de tipo referencia?

\begin{verbatim}
This is [an example][id] reference-style link.

Then, anywhere in the document, you define your link label like this, on a line by itself:

[id]: http://example.com/  "Optional Title Here"
\end{verbatim}
\item
  ¿Cómo se pone una imagen?

\begin{verbatim}
![Screenshot of the toolbar](http://so.mrozekma.com/editor-bar-help-button.png)
\end{verbatim}
\item
  ¿Cómo se anidan listas?

\begin{verbatim}
 To put other Markdown blocks in a list; just indent four spaces for each nesting level
  For example 

  1. Dog
      1. German Shepherd
      2. Belgian Shepherd
          1. Malinois
          2. Groenendael
          3. Tervuren
  2. Cat
      1. Siberian
      2. Siamese
\end{verbatim}
\item
  ¿Cómo se inserta una línea horizontal?
\end{itemize}


\section{GitBook}
\subsection{Preguntas de Repaso de
GitBook}\label{preguntas-de-repaso-de-gitbook}

\begin{enumerate}
\def\labelenumi{\arabic{enumi}.}
\itemsep1pt\parskip0pt\parsep0pt
\item
  ¿Cómo se escribe en GitBook esta fórmula?
\end{enumerate}

\[x=\frac{1+y}{1+2z^2}\]

\begin{enumerate}
\def\labelenumi{\arabic{enumi}.}
\itemsep1pt\parskip0pt\parsep0pt
\item
  ¿Como se escribe un código JavaScript de manera que se muestre con
  \emph{syntax highlingting}?
\item
  ¿Como se escribe un bloque de cita \emph{blokquote}?
\item
  ¿Que pasos debo dar para insertar un vídeo de YouTube en mi libro
  GitBook?
\item
  Escriba el código MD para producir una tabla como esta:
\end{enumerate}

\begin{longtable}[c]{@{}ll@{}}
\toprule
First Header & Second Header\tabularnewline
\midrule
\endhead
Content Cell & Content Cell\tabularnewline
Content Cell & Content Cell\tabularnewline
\bottomrule
\end{longtable}

\begin{enumerate}
\def\labelenumi{\arabic{enumi}.}
\itemsep1pt\parskip0pt\parsep0pt
\item
  ¿Donde puedo encontrar la URL del repositorio en GitBook del libro?
\item
  Explique los pasos para publicar un libro GitBook en GitHub usando las
  gh-pages de GitHub manualmente
\item
  ¿Que atributo debo de poner en \texttt{book.json} para alojar los
  Markdown del libro en un directorio distinto del raiz?
\item
  Explique como instalar y usar Gitbook
\item
  Como se despliega un libro en GitHub
\item
  Como se despliega un libro en \texttt{gitbook.com}
\item
  Como se despliega un libro en Heroku
\item
  Como se despliega un libro en una máquina virtual de
  \texttt{iaas.ull.es}
\end{enumerate}


\section{gulp}
\subsection{Preguntas de Repaso de
Gulp}\label{preguntas-de-repaso-de-gulp}

\begin{enumerate}
\def\labelenumi{\arabic{enumi}.}
\itemsep1pt\parskip0pt\parsep0pt
\item
  Complete las partes que faltan del siguiente \texttt{gulpfile.js} en
  el que se lleva a cabo una tarea para la optimización (uglify/minify)
  de la aplicación de la práctica de la temperatura:
\end{enumerate}

\begin{Shaded}
\begin{Highlighting}[]
\OtherTok{/tmp/pl}\NormalTok{-grado-temperature-}\FunctionTok{converter}\NormalTok{(karma)]$ cat }\OtherTok{gulpfile}\NormalTok{.}\FunctionTok{js}
\KeywordTok{var} \NormalTok{gulp    = }\FunctionTok{require}\NormalTok{(}\StringTok{'gulp'}\NormalTok{),}
    \NormalTok{gutil   = }\FunctionTok{require}\NormalTok{(}\StringTok{'gulp-util'}\NormalTok{),}
    \NormalTok{uglify  = }\FunctionTok{require}\NormalTok{(}\StringTok{'gulp-uglify'}\NormalTok{),}
    \NormalTok{concat  = }\FunctionTok{require}\NormalTok{(}\StringTok{'gulp-concat'}\NormalTok{);}
\KeywordTok{var} \NormalTok{minifyHTML = }\FunctionTok{require}\NormalTok{(}\StringTok{'gulp-minify-html'}\NormalTok{);}
\KeywordTok{var} \NormalTok{minifyCSS  = }\FunctionTok{require}\NormalTok{(}\StringTok{'gulp-minify-css'}\NormalTok{);}

\OtherTok{gulp}\NormalTok{.}\FunctionTok{____}\NormalTok{(}\StringTok{'minify'}\NormalTok{, }\KeywordTok{function} \NormalTok{() \{}
  \OtherTok{gulp}\NormalTok{.}\FunctionTok{___}\NormalTok{(}\StringTok{'temperature.js'}\NormalTok{)}
  \NormalTok{.}\FunctionTok{____}\NormalTok{(}\FunctionTok{uglify}\NormalTok{())}
  \NormalTok{.}\FunctionTok{___}\NormalTok{(}\OtherTok{gulp}\NormalTok{.}\FunctionTok{____}\NormalTok{(}\StringTok{'minified'}\NormalTok{));}

  \OtherTok{gulp}\NormalTok{.}\FunctionTok{__}\NormalTok{(}\StringTok{'./index.html'}\NormalTok{)}
    \NormalTok{.}\FunctionTok{___}\NormalTok{(}\FunctionTok{minifyHTML}\NormalTok{())}
    \NormalTok{.}\FunctionTok{___}\NormalTok{(}\OtherTok{gulp}\NormalTok{.}\FunctionTok{___}\NormalTok{(}\StringTok{'./minified/'}\NormalTok{))}

  \OtherTok{gulp}\NormalTok{.}\FunctionTok{__}\NormalTok{(}\StringTok{'./*.css'}\NormalTok{)}
   \NormalTok{.}\FunctionTok{___}\NormalTok{(}\FunctionTok{minifyCSS}\NormalTok{(\{}\DataTypeTok{keepBreaks}\NormalTok{:}\KeywordTok{true}\NormalTok{\}))}
   \NormalTok{.}\FunctionTok{___}\NormalTok{(}\OtherTok{gulp}\NormalTok{.}\FunctionTok{___}\NormalTok{(}\StringTok{'./minified/'}\NormalTok{))}
        \NormalTok{\});}
\end{Highlighting}
\end{Shaded}

\begin{itemize}
\itemsep1pt\parskip0pt\parsep0pt
\item
  Explique los pasos para publicar un libro GitBook en GitHub usando
  \texttt{gulp}
\item
  Explique los pasos para actualizar automáticamente los HTML del libro
  GitBook en su máquina virtual del iaas usando \texttt{gulp}
\end{itemize}


\section{npm}
\subsection{Preguntas de Repaso de npm y
package.json}\label{preguntas-de-repaso-de-npm-y-package.json}

\begin{enumerate}
\def\labelenumi{\arabic{enumi}.}
\item
  ¿Con que comando creo el fichero \texttt{package.json}?
\item
  Explique en consiste el versionado semántico/semantic versioning.
  ¿Cual es el nombre en inglés de los tres números de version? ¿Como
  cambian?
\item
  ¿Que se guarda en el campo \texttt{"dependencies": \{\}} de
  \texttt{package.json}?
\item
  ¿Que opción debo añadir al comando \texttt{npm install} para que la
  librería instalada se añada como dependencia en el fichero
  \texttt{package.json}?
\item
  ¿Que se guarda en el campo \texttt{"devDependencies": \{\}} de
  \texttt{package.json}?
\item
  ¿Que opción debo añadir al comando \texttt{npm install} para que la
  librería instalada se añada como \texttt{"devDependencies"} en el
  fichero \texttt{package.json}?
\item
  Explique que significan en los objetos que describen las dependencias
  dentro \texttt{package.json} las siguientes notaciones:

  \begin{enumerate}
  \def\labelenumii{\arabic{enumii}.}
  \item
    \texttt{*}
  \item
    \texttt{latest}
  \end{enumerate}
\item
  ¿Cómo instalo una versión anterior de un package npm?
\item
  \href{http://stackoverflow.com/questions/10972176/find-the-version-of-an-installed-npm-package}{¿Cómo
  encuentro la versión de un paquete NodeJS instalado?}
\end{enumerate}


\section{Heroku}
\subsection{Preguntas de Repaso de
Heroku}\label{preguntas-de-repaso-de-heroku}

\begin{enumerate}
\def\labelenumi{\arabic{enumi}.}
\item
  Una vez instalado el Heroku cli nos debemos autenticar en heroku con
  el cliente. ¿Cual es el comando para autenticarnos?
\item
  ¿Cual es el comando para crear nuestra aplicación en Heroku (suponemos
  que ya esta bajo el control de \texttt{git}? ¿Qué remoto tendremos
  definido después de crear nuestra aplicación en Heroku?
\item
  ¿Cual es el comando para desplegar nuestra aplicación en Heroku?
\item
  Si la versión que queremos publicar en heroku no está en la rama
  \texttt{master} sino que está en la rama \texttt{tutu} ¿Como debemos
  modificar el comando anterior?
\item
  ¿Con que comando puedo abrir una ventana en el navegador que visite la
  aplicación desplegada? ¿Que formato tiene la URL para nuestra
  aplicación?
\item
  ¿Con que comando puedo ver los logs de la aplicación desplegada?
\item
  ¿Como se debe llamar el fichero en el que explicito que comando debe
  usarse para arrancar nuestra aplicación en Heroku?
\item
  Heroku se percata que nuestra aplicación es una aplicación
  desarrollada con \texttt{Node.js} por la presencia de un cierto
  fichero. ¿De que fichero estamos hablando?
\item
  ¿Cual es la mejor forma de ejecutar en local una aplicación express.js
  que va a ser desplegada en Heroku?
\item
  Explique los pasos para desplegar una aplicación en Heroku
\item
  Explique como resolver los problemas que pueden surgir cuando la
  aplicación desplegada en Heroku no funciona correctamente
\item
  \href{../recursos/heroku.md}{¿Como consulto el token para hacer uso de
  la API de Heroku?}
\item
  \href{../recursos/heroku.md}{¿Cómo creo una app en Heroku usando la
  API de Heroku?}
\end{enumerate}

\begin{itemize}
\item
  \href{https://devcenter.heroku.com/articles/setting-up-apps-using-the-heroku-platform-api\#creating-an-app-setup}{Véase}
\item
  With a source tarball, which contains an app.json, call the
  https://api.heroku.com/app-setups endpoint to setup the app.json
  enabled application on Heroku. The request body must contain a source
  URL that points to the tarball of your application's source code.
\item
  Let's use cURL to call the app-setups endpoint:

\begin{verbatim}
$ curl -n -X POST https://api.heroku.com/app-setups \
-H "Content-Type:application/json" \
-H "Accept:application/vnd.heroku+json; version=3" \
-d '{"source_blob": { "url":"https://github.com/heroku/ruby-rails-sample/tarball/master/"} }'
\end{verbatim}
\item
  Explique los pasos para publicar un libro GitBook en Heroku
\end{itemize}


%\subsection{Preguntas de REST y Servicios
Web}\label{preguntas-de-rest-y-servicios-web}

\begin{enumerate}
\def\labelenumi{\arabic{enumi}.}
\itemsep1pt\parskip0pt\parsep0pt
\item
  Defina que es un servicio web
\item
  Explique que es REST
\end{enumerate}


\section{ssh}
\subsection{Preguntas de SSH}\label{preguntas-de-ssh}

\begin{enumerate}
\def\labelenumi{\arabic{enumi}.}
\itemsep1pt\parskip0pt\parsep0pt
\item
  Explique como se generan las claves privada y pública
\item
  Como se publica una clave?
\item
  Indique como se puede configurar el cliente SSH para simplificar la
  conexión
\item
  ¿Cómo puedo ejecutar un script en una máquina accesible via SSH?
\end{enumerate}


\section{Rutas en express}
\subsection{Rutas en Express}\label{rutas-en-express}

\begin{enumerate}
\def\labelenumi{\arabic{enumi}.}
\itemsep1pt\parskip0pt\parsep0pt
\item
  Escriba un servidor que sirva ficheros estáticos desde el directorio
  \texttt{/public}
\item
  El servidor deberá responder a requests \texttt{GET} en las rutas
  \texttt{/user/nombredeusuario} (donde \texttt{nombredeusuario} varía)
  con una página que diga \texttt{Hola nombredeusuario} usando el método
  \texttt{render}del objeto \texttt{response}
\end{enumerate}

\begin{itemize}
\itemsep1pt\parskip0pt\parsep0pt
\item
  La página se elaborara con una vista que debe estar en el directorio
  \texttt{views/} usando el motor de vistas \texttt{ejs}
\item
  La página elaborada en la respuesta tendrá un tag \texttt{img}
  referenciando a una imagen que está en \texttt{public/}
\end{itemize}

\begin{enumerate}
\def\labelenumi{\arabic{enumi}.}
\setcounter{enumi}{2}
\itemsep1pt\parskip0pt\parsep0pt
\item
  Escriba un middleware que intercepte en las rutas
  \texttt{/user/nombredeusuario} y que vuelque en la consola información
  sobre el \href{https://expressjs.com/en/4x/api.html\#req}{request}:
  (por ejemplo los atributos \texttt{method}, \texttt{path}, etc.)
\item
  Explique como se puede aislar el código anterior en un fichero
  \texttt{routes/user.js} que sea cargado desde el programa principal
\item
  Explique que hay que hacer para desplegar la aplicación en Heroku
\item
  Explique que hay que hacer para desplegar la aplicación en la máquina
  virtual del iaas
\end{enumerate}


%\section{HTTPS}
%\subsection{Preguntas de HTTPS}\label{preguntas-de-https}

\begin{itemize}
\itemsep1pt\parskip0pt\parsep0pt
\item
  ¿Cuales son las dos funcionalidades principales proveídas por la capa
  SSL?
\item
  Verifying that you are talking directly to the server that you think
  you are talking to
\item
  Ensuring that only the server can read what you send it and only you
  can read what it sends back
\item
  ¿Es posible que alguien intercepte un mensaje utilizando HTTPS?
\item
  The really, really clever part is that \textbf{anyone can intercept
  every single one of the messages you exchange with a server, including
  the ones where you are agreeing on the key and encryption strategy to
  use, and still not be able to read any of the actual data you send.}
\item
  ¿Cuales son los tres objetivos de la fase de \emph{handshake} entre un
  cliente y un servidor utilizando TLS?
\item
  To satisfy the client that it is talking to the right server (and
  optionally visa versa)
\item
  For the parties to have agreed on a
  \emph{\href{https://en.wikipedia.org/wiki/Cipher_suite}{cipher
  suite}}, which includes which encryption algorithm they will use to
  exchange data
\item
  For the parties to have agreed on any necessary keys for this
  algorithm
\item
  ¿Como se llaman las tres fases en las que se puede descomponer la
  etapa de \href{http://www.dictionary.com/browse/handshake}{handshake}?
\item
  Hello, Certificate Exchange and Key Exchange*
\item
  Describa la primera fase del \emph{handshake}
\item
  The \href{http://www.dictionary.com/browse/handshake}{handshake}
  begins with the client sending a \texttt{ClientHello} message.
\item
  This contains all the information the server needs in order to connect
  to the client via SSL, including

  \begin{itemize}
  \itemsep1pt\parskip0pt\parsep0pt
  \item
    the various cipher suites
  \item
    and maximum SSL version that it supports.
  \end{itemize}
\item
  The server responds with a \texttt{ServerHello}, which contains
  similar information required by the client, including

  \begin{itemize}
  \itemsep1pt\parskip0pt\parsep0pt
  \item
    a decision based on the client's preferences about which cipher
    suite and version of SSL will be used.
  \end{itemize}
\item
  Describa la segunda fase del \emph{handshake}
\item
  Now that contact has been established, the server has to prove its
  identity to the client.
\item
  This is achieved using its SSL certificate, which is a very tiny bit
  like its passport.
\item
  An SSL certificate contains various pieces of data, including the

  \begin{itemize}
  \itemsep1pt\parskip0pt\parsep0pt
  \item
    name of the owner,
  \item
    the property (eg. domain) it is attached to,
  \item
    the certificate's public key,
  \item
    the \href{https://en.wikipedia.org/wiki/Digital_signature}{digital
    signature} and
  \item
    information about the certificate's validity dates.
  \end{itemize}
\item
  The client checks that it either

  \begin{itemize}
  \itemsep1pt\parskip0pt\parsep0pt
  \item
    implicitly trusts the certificate,
  \item
    or that it is verified and trusted by one of several Certificate
    Authorities (CAs) that it also implicitly trusts.
  \end{itemize}
\item
  Note that the server is also allowed to require a certificate to prove
  the client's identity, but this typically only happens in very
  sensitive applications.
\item
  Describa la tercera fase del \emph{handshake}
\item
  The encryption of the actual message data exchanged by the client and
  server will be done using a symmetric algorithm, the exact nature of
  which was already agreed during the \textbf{Hello phase}.
\item
  A \textbf{symmetric algorithm} uses a single key for both encryption
  and decryption, in contrast to asymmetric algorithms that require a
  public/private key pair.
\item
  Both parties need to agree on this single, symmetric key, a process
  that is accomplished securely using asymmetric encryption and the
  server's public/private keys.
\item
  The client generates a random key to be used for the main, symmetric
  algorithm.

  \begin{itemize}
  \itemsep1pt\parskip0pt\parsep0pt
  \item
    It encrypts it using an algorithm also agreed upon during the Hello
    phase, and the server's public key (found on its SSL certificate).
  \item
    It sends this encrypted key to the server, where it is decrypted
    using the server's private key, and the interesting parts of the
    \href{http://www.dictionary.com/browse/handshake}{handshake} are
    complete.
  \end{itemize}
\item
  ¿Que tipo de cifrado se utiliza una vez que a finalizado con éxito la
  fase de handshake?
\item
  The parties are sufficiently happy that they are talking to the right
  person, and have secretly agreed on a key to symmetrically encrypt the
  data that they are about to send each other.
\item
  ¿Cuales son las dos razones por las que podríamos confiar en un
  certificado SSL?
\item
  There are 2 sensible reasons why you might trust a certificate:

  \begin{itemize}
  \itemsep1pt\parskip0pt\parsep0pt
  \item
    If it's on a list of certificates that you trust implicitly
  \item
    If it's able to prove that it is trusted by the controller of one of
    the certificates on the above list
  \item
    The first criteria is easy to check. Your browser has a
    pre-installed list of trusted SSL certificates from Certificate
    Authorities (CAs) that you can view, add and remove from.
  \item
    These certificates are controlled by a centralised group of (in
    theory, and generally in practice) extremely secure, reliable and
    trustworthy organisations like

    \begin{itemize}
    \itemsep1pt\parskip0pt\parsep0pt
    \item
      \href{https://letsencrypt.org/}{Let's Encrypt} (Let's Encrypt is a
      free, automated, and open Certificate Authority),
    \item
      \href{http://www.cacert.org/}{CAcert.org es una autoridad
      certificadora dirigida por la comunidad que emite certificados
      gratuitos al público}
    \item
      Symantec,
    \item
      Comodo and
    \item
      GoDaddy.
    \end{itemize}
  \end{itemize}
\item
  Describa como funciona una firma digital
\item
  As already noted, SSL certificates have an associated public/private
  key pair

  \begin{itemize}
  \itemsep1pt\parskip0pt\parsep0pt
  \item
    The public key is distributed as part of the certificate, and the
    private key is kept incredibly safely guarded
  \item
    This pair of asymmetric keys is used in the SSL
    \href{http://www.dictionary.com/browse/handshake}{handshake} to
    exchange a further key for both parties to symmetrically encrypt and
    decrypt data
  \item
    \textbf{The client uses the server's public key to encrypt the
    symmetric key and send it securely to the server, and the server
    uses its private key to decrypt it}
  \item
    \includegraphics{https://raviranjankr.files.wordpress.com/2012/08/asymmetric-encryption.gif}
  \item
    Anyone can encrypt using the public key, but only the server can
    decrypt using the private key
  \end{itemize}
\item
  The opposite is true for a digital signature.

  \begin{itemize}
  \itemsep1pt\parskip0pt\parsep0pt
  \item
    A certificate can be \emph{``signed''} by another authority,
    \href{https://www.google.es/webhp?sourceid=chrome-instant\&ion=1\&espv=2\&ie=UTF-8\#q=define\%20whereby}{whereby}
    the authority effectively goes on record as saying
  \end{itemize}

  \emph{``We have verified that the controller of this certificate also
  controls the property (domain) listed on the certificate''}.

  \begin{itemize}
  \itemsep1pt\parskip0pt\parsep0pt
  \item
    In this case the authority uses their private key to (broadly
    speaking) encrypt the contents of the certificate, and this cipher
    text is attached to the certificate as its digital signature.
  \item
    Anyone can decrypt this signature using the authority's public key,
    and verify that it results in the expected decrypted value.
  \item
    But only the authority can encrypt content using the private key,
    and so only the authority can actually create a valid signature in
    the first place.
  \end{itemize}
\item
  So if a server comes along claiming to have a certificate for
  Microsoft.com that is signed by Symantec (or some other CA), your
  browser doesn't have to take its word for it.

  \begin{itemize}
  \itemsep1pt\parskip0pt\parsep0pt
  \item
    If it is legit, Symantec will have used their (ultra-secret) private
    key to generate the server's SSL certificate's digital signature,
    and so your browser use can use their (ultra-public) public key to
    check that this signature is valid.
  \item
    Symantec will have taken steps to ensure the organisation they are
    signing for really does own Microsoft.com, and so given that your
    client trusts Symantec, it can be sure that it really is talking to
    Microsoft Inc.
    \includegraphics{http://www.hill2dot0.com/wiki/images/f/ff/Digital_Signature.jpg}
  \end{itemize}
\item
  Pueden en un coffee shop conocer los contenidos de mi tráfico HTTPS
  desde mi portátil sobre su red?
\item
  Nope.

  \begin{itemize}
  \itemsep1pt\parskip0pt\parsep0pt
  \item
    The magic of public-key cryptography means that an attacker can
    watch every single byte of data exchanged between your client and
    the server and still have no idea what you are saying to each other
    beyond roughly how much data you are exchanging.
  \item
    However, your normal HTTP traffic is still very vulnerable on an
    insecure wi-fi network, and a flimsy website can fall victim to any
    number of workarounds that somehow trick you into sending HTTPS
    traffic either over plain HTTP or just to the wrong place
    completely.
  \item
    For example, even if a login form submits a username/password combo
    over HTTPS, if the form itself is loaded insecurely over HTTP then
    an attacker could intercept the form's HTML on its way to your
    machine and modify it to send the login details to their own
    endpoint.
  \end{itemize}
\item
  Puede mi empresa conocer los contenidos de mi tráfico HTTPS sobre la
  red cuando uso la máquina que me proveen?
\item
  If you are also using a machine controlled by your company, then yes.

  \begin{itemize}
  \itemsep1pt\parskip0pt\parsep0pt
  \item
    Remember that at the root of every chain of trust lies an implicitly
    trusted CA, and that a list of these authorities is stored in your
    browser.
  \item
    Your company could use their access to your machine to \textbf{add
    their own self-signed certificate to this list of CAs}.
  \item
    They could then intercept all of your HTTPS requests, presenting
    certificates claiming to represent the appropriate website, signed
    by their fake-CA and therefore unquestioningly trusted by your
    browser.
  \item
    Since you would be encrypting all of your HTTPS requests using their
    dodgy certificate's public key, they could use the corresponding
    private key to decrypt and inspect (even modify) your request, and
    then send it onto it's intended location.
  \item
    They probably don't. But they could.
  \end{itemize}
\item
  Incidentally, this is also how you use a proxy to inspect and modify
  the otherwise inaccessible
  \href{http://nickfishman.com/post/50557873036/reverse-engineering-native-apps-by-intercepting-network}{HTTPS
  requests made by an iPhone app}.
\end{itemize}


%\section{Passport}
%\subsection{Preguntas de Passport}\label{preguntas-de-passport}

\begin{itemize}
\itemsep1pt\parskip0pt\parsep0pt
\item
  ¿Que es OAuth?
\item
  OAuth provides a method for users to grant third-party limited access
  (in scope, duration, etc.) access to their resources without sharing
  their passwords
\item
  ¿Quienes son los cuatro roles que aparecen en una autenticación con
  OAuth?
\end{itemize}

\begin{enumerate}
\def\labelenumi{\arabic{enumi}.}
\itemsep1pt\parskip0pt\parsep0pt
\item
  resource owner: An entity capable of granting access to a protected
  resource. When the resource owner is a person, it is referred to as an
  end-user. (El usuario)
\item
  resource server: The server hosting the protected resources, capable
  of accepting and responding to protected resource requests using
  access tokens. (El servidor de Pinterest)
\item
  client: An application making protected resource requests on behalf of
  the resource owner and with its authorization (por ejemplo, un cliente
  de pinterest en el teléfono). The term " client" does not imply any
  particular implementation characteristics (e.g., whether the
  application executes on a server, a desktop, or other devices).
\item
  authorization server: The server issuing access tokens to the client
  after successfully authenticating the resource owner and obtaining
  authorization (por ejemplo, Facebook, cuando nos autenticamos con
  Facebook)
\end{enumerate}

\begin{itemize}
\itemsep1pt\parskip0pt\parsep0pt
\item
  ¿Qué tres elementos de información suelen ser necesarios a la hora de
  registrar nuestra aplicación ante un proveedor de OAuth?
\item
  Before using OAuth with your application, you must register your
  application with the service.
\item
  This is done through a registration form in the developer or API
  portion of the service's website, where you will provide the following
  information (and probably details about your application):

  \begin{enumerate}
  \def\labelenumi{\arabic{enumi}.}
  \itemsep1pt\parskip0pt\parsep0pt
  \item
    Application Name
  \item
    Application Website
  \item
    Redirect URI or Callback URL
  \end{enumerate}
\item
  ¿Que se debe poner en \emph{Redirect URI or Callback URL} cuando se
  está registrando nuestra aplicación?
\item
  The redirect URI is where the service will redirect the user after
  they authorize (or deny) your application, and therefore the part of
  your application that will handle authorization codes or access
  tokens.
\item
  Una vez que registramos la aplicación, el servicio provee las
  credenciales del cliente. ¿En que consisten esas credenciales?
\item
  Once your application is registered, the service will issue client
  credentials in the form of a client identifier and a client secret.
\item
  The Client ID is a publicly exposed string that is used by the service
  API to identify the application, and is also used to build
  authorization URLs that are presented to users.
\item
  The Client Secret is used to authenticate the identity of the
  application to the service API when the application requests to access
  a user's account, and must be kept private between the application and
  the API.
\item
  ¿Que es \emph{passport}, que funcionalidades provee y como funciona?
\item
  Passport is authentication middleware for Node.js. Extremely flexible
  and modular, Passport can be unobtrusively dropped in to any
  Express-based web application. A comprehensive set of strategies
  support authentication using a username and password, Facebook,
  Twitter, and more.
\item
  Rellene las partes que faltan:
\end{itemize}

```javascript var passport = require(`passport'); var Strategy =
require(`\_\_\_\_\_\_\_\_\_\_\_\_\_\_\_').Strategy; var github =
require(`octonode'); \ldots{}. var datos\_config =
JSON.parse(JSON.stringify(config));

passport.use(new Strategy(\{ clientID: datos\_config.clientID,
clientSecret: datos\_config.clientSecret, callbackURL: callbackURL\_ \},
function(accessToken, refreshToken, profile, cb) \{

\begin{verbatim}
    var token = datos_config.token;
    var client = github.client(_____);

    var ghorg = client.___('ULL-ESIT-SYTW-1617');

    ghorg.______(profile.username, (err,result) =>
    {
        if(err) console.log(err);
        console.log("Result:"+result);
        if(result == true)
          return cb(null, profile);
        else {
          return cb(null,null);
        }
    });
\end{verbatim}

\})); ``` - Respuesta:

```javascript var passport = require(`passport'); var Strategy =
require(`passport-github').Strategy; var github = require(`octonode');
\ldots{}. var datos\_config = JSON.parse(JSON.stringify(config));

passport.use(new Strategy(\{ clientID: datos\_config.clientID,
clientSecret: datos\_config.clientSecret, callbackURL: callbackURL\_ \},
function(accessToken, refreshToken, profile, cb) \{

\begin{verbatim}
    var token = datos_config.token;
    var client = github.client(token);

    var ghorg = client.org('ULL-ESIT-SYTW-1617');

    ghorg.member(profile.username, (err,result) =>
    {
        if(err) console.log(err);
        console.log("Result:"+result);
        if(result == true)
          return cb(null, profile);
        else {
          return cb(null,null);
        }
    });
  // return cb(null, profile);
\end{verbatim}

\})); ```


\end{document}
