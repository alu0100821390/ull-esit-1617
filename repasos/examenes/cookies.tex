\subsection{Preguntas de Repaso de
Cookies}\label{preguntas-de-repaso-de-cookies}

\begin{enumerate}
\def\labelenumi{\arabic{enumi}.}
\itemsep1pt\parskip0pt\parsep0pt
\item
  Defina y explique el concepto de Cookie
\item
  Defina y explique el concepto de Cookie para la Gestión de una Sesión
\item
  Defina y explique el concepto de Cookie para la Autenticación en un
  Website
\item
  ¿Que es un secure cookie? (cookie seguro)
\end{enumerate}

A secure cookie has the secure attribute enabled and is only used via
HTTPS, ensuring that the cookie is always encrypted when transmitting
from client to server. This makes the cookie less likely to be exposed
to cookie theft via eavesdropping. 4. ¿Como se llama el middleware
express que me facilita el manejo de los cookies? ¿Cual es el código
para poner en funcionamiento dicho middleware?

\begin{verbatim}
% cookie-parser
% var cookieParser = require('cookie-parser');
% app.use(cookieParser());
\end{verbatim}

\begin{enumerate}
\def\labelenumi{\arabic{enumi}.}
\setcounter{enumi}{4}
\item
  Dentro de un middleware express ¿Cómo se llama el método del objeto
  \texttt{res} que me permite establecer un cookie? ¿Que argumentos
  recibe?

\begin{verbatim}
%   res.cookie('cookie_name', 'cookie_value', {expire : new Date() + 9999}).send(
%      "Cookie is set: goto to browser's console and write document.cookie.");
\end{verbatim}
\item
  ¿Que método del objeto \texttt{request} me permite ver los cookies
  establecidos?

\begin{verbatim}
console.log("Cookies :  ", req.cookies);
\end{verbatim}
\item
  ¿Como borro un cookie en el servidor express?

\begin{verbatim}
res.clearCookie('cookie_name');
\end{verbatim}
\item
  ¿Que es la Query String? Ponga un ejemplo de Query String
\end{enumerate}

On the World Wide Web, a query string is the part of a uniform resource
locator (URL) containing data that does not fit conveniently into a
hierarchical path structure. The query string commonly includes fields
added to a base URL by a Web browser or other client application, for
example as part of an HTML form. A typical URL containing a query string
is as follows:

\begin{verbatim}
http://example.com/over/there?name=ferret
\end{verbatim}

\begin{enumerate}
\def\labelenumi{\arabic{enumi}.}
\setcounter{enumi}{8}
\item
  If a form is embedded in an HTML page as follows:

\begin{Shaded}
\begin{Highlighting}[]
\KeywordTok{<form}\OtherTok{ action=}\StringTok{"/hello"}\OtherTok{ method=}\StringTok{"get"}\KeywordTok{>}
  \KeywordTok{<input}\OtherTok{ type=}\StringTok{"text"}\OtherTok{ name=}\StringTok{"first"} \KeywordTok{/>}
  \KeywordTok{<input}\OtherTok{ type=}\StringTok{"text"}\OtherTok{ name=}\StringTok{"second"} \KeywordTok{/>}
  \KeywordTok{<input}\OtherTok{ type=}\StringTok{"submit"} \KeywordTok{/>}
\KeywordTok{</form>}
\end{Highlighting}
\end{Shaded}

  and the user inserts the strings \texttt{this is a field} and
  \texttt{was it clear already} in the two text fields and presses the
  \texttt{submit} button, the handler of the route \texttt{/hello} (the
  route specified by the \texttt{action} attribute of the form element
  in the above example) will receive a query string. Write down the
  query string it receives
\end{enumerate}

\begin{verbatim}
first=this+is+a+field&second=was+it+clear+already
\end{verbatim}
