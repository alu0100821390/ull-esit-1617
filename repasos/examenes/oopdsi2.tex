\subsection{Preguntas de Programación Orientada a
Objetos}\label{preguntas-de-programaciuxf3n-orientada-a-objetos}

\begin{enumerate}
\def\labelenumi{\arabic{enumi}.}
\item
  Escriba un código JavaScript que defina una clase \texttt{Persona} con
  atributos \texttt{nombre} y \texttt{apellidos} y que disponga de un
  método \texttt{saluda}.
\item
  Escriba una clase \texttt{Programador} que hereda de \texttt{Persona}
\item
  Escriba un código ECMA6 que defina una clase \texttt{Persona} con
  atributos \texttt{nombre} y \texttt{apellidos} y que disponga de un
  método \texttt{saluda}.
\item
  Usando ECMA6 escriba una clase \texttt{Programador} que hereda de
  \texttt{Persona}
\item
  Explique las diferencias en la salida entre este código
\end{enumerate}

\begin{verbatim}
      function Person() {
        this.age = 0;

        function growUp() {
          this.age += 10;
        }
        growUp();
        console.log(this.age);
      }
      var p = new Person();
\end{verbatim}

y este otro:

\begin{verbatim}
      function Person() {
        this.age = 0;

        var growUp = () => {
          this.age += 10;
        }
        growUp();
        console.log(this.age);
      }

      var p = new Person();
\end{verbatim}

Justifique su respuesta.

\begin{enumerate}
\def\labelenumi{\arabic{enumi}.}
\setcounter{enumi}{1}
\item
  Explique que hacen los métodos \texttt{bind}, \texttt{apply} y
  \texttt{call} y cuales son sus similitudes y diferencias
\item
  ¿Cual es el significado del primer argumento del método
  \texttt{Object.cretate}? ¿Y el segundo?
\item
  Todo objeto JavaScript tiene una propiedad \texttt{"prototype"}
  ¿verdadero o falso?
\item
  La propiedad \texttt{prototype} de una función es un objeto de tipo
  \texttt{Function} ¿verdadero o falso?
\item
  El \texttt{prototype} de una función es un objeto de tipo
  \texttt{Function} ¿verdadero o falso?
\item
  ¿Cual es el problema con este código? ¿Como se arregla el problema?

\begin{verbatim}
Object.prototype.nonsense = "hi";
for (var name in map)
  console.log(name);
\end{verbatim}
\item
  ¿Que significa que una propiedad es no-enumerable? 
\item
  ¿Como puedo crear un objeto que carezca de prototipo? 
\end{enumerate}

10 El argumento \texttt{descriptor} del método

\begin{verbatim}
Object.defineProperty(obj, prop, descriptor)
\end{verbatim}

puede ser de uno de dos tipos: \textbf{data descriptors} o
\textbf{accessor descriptors}.

\begin{itemize}
\itemsep1pt\parskip0pt\parsep0pt
\item
  Un \texttt{data descriptor} es una propiedad que tiene un
  \texttt{value}, que puede o no ser \texttt{writable}.
\item
  Un \texttt{accessor descriptor} es una propiedad que describe un par
  de funciones getter-setter.
\end{itemize}

Un descriptor puede ser de uno de estos tipos pero no puede ser ambos.

Explique cuales de estas propiedades pertenecen a que tipo, cual es su
valor por defecto y que describen:

\begin{enumerate}
\def\labelenumi{\arabic{enumi}.}
\itemsep1pt\parskip0pt\parsep0pt
\item
  \texttt{configurable} 
\item
  \texttt{enumerable} 
\item
  \texttt{value} 
\item
  \texttt{writable} 
\item
  \texttt{get} 
\item
  \texttt{set} 
\end{enumerate}
