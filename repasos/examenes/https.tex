\subsection{Preguntas de HTTPS}\label{preguntas-de-https}

\begin{itemize}
\itemsep1pt\parskip0pt\parsep0pt
\item
  ¿Cuales son las dos funcionalidades principales proveídas por la capa
  SSL?
\item
  Verifying that you are talking directly to the server that you think
  you are talking to
\item
  Ensuring that only the server can read what you send it and only you
  can read what it sends back
\item
  ¿Es posible que alguien intercepte un mensaje utilizando HTTPS?
\item
  The really, really clever part is that \textbf{anyone can intercept
  every single one of the messages you exchange with a server, including
  the ones where you are agreeing on the key and encryption strategy to
  use, and still not be able to read any of the actual data you send.}
\item
  ¿Cuales son los tres objetivos de la fase de \emph{handshake} entre un
  cliente y un servidor utilizando TLS?
\item
  To satisfy the client that it is talking to the right server (and
  optionally visa versa)
\item
  For the parties to have agreed on a
  \emph{\href{https://en.wikipedia.org/wiki/Cipher_suite}{cipher
  suite}}, which includes which encryption algorithm they will use to
  exchange data
\item
  For the parties to have agreed on any necessary keys for this
  algorithm
\item
  ¿Como se llaman las tres fases en las que se puede descomponer la
  etapa de \href{http://www.dictionary.com/browse/handshake}{handshake}?
\item
  Hello, Certificate Exchange and Key Exchange*
\item
  Describa la primera fase del \emph{handshake}
\item
  The \href{http://www.dictionary.com/browse/handshake}{handshake}
  begins with the client sending a \texttt{ClientHello} message.
\item
  This contains all the information the server needs in order to connect
  to the client via SSL, including

  \begin{itemize}
  \itemsep1pt\parskip0pt\parsep0pt
  \item
    the various cipher suites
  \item
    and maximum SSL version that it supports.
  \end{itemize}
\item
  The server responds with a \texttt{ServerHello}, which contains
  similar information required by the client, including

  \begin{itemize}
  \itemsep1pt\parskip0pt\parsep0pt
  \item
    a decision based on the client's preferences about which cipher
    suite and version of SSL will be used.
  \end{itemize}
\item
  Describa la segunda fase del \emph{handshake}
\item
  Now that contact has been established, the server has to prove its
  identity to the client.
\item
  This is achieved using its SSL certificate, which is a very tiny bit
  like its passport.
\item
  An SSL certificate contains various pieces of data, including the

  \begin{itemize}
  \itemsep1pt\parskip0pt\parsep0pt
  \item
    name of the owner,
  \item
    the property (eg. domain) it is attached to,
  \item
    the certificate's public key,
  \item
    the \href{https://en.wikipedia.org/wiki/Digital_signature}{digital
    signature} and
  \item
    information about the certificate's validity dates.
  \end{itemize}
\item
  The client checks that it either

  \begin{itemize}
  \itemsep1pt\parskip0pt\parsep0pt
  \item
    implicitly trusts the certificate,
  \item
    or that it is verified and trusted by one of several Certificate
    Authorities (CAs) that it also implicitly trusts.
  \end{itemize}
\item
  Note that the server is also allowed to require a certificate to prove
  the client's identity, but this typically only happens in very
  sensitive applications.
\item
  Describa la tercera fase del \emph{handshake}
\item
  The encryption of the actual message data exchanged by the client and
  server will be done using a symmetric algorithm, the exact nature of
  which was already agreed during the \textbf{Hello phase}.
\item
  A \textbf{symmetric algorithm} uses a single key for both encryption
  and decryption, in contrast to asymmetric algorithms that require a
  public/private key pair.
\item
  Both parties need to agree on this single, symmetric key, a process
  that is accomplished securely using asymmetric encryption and the
  server's public/private keys.
\item
  The client generates a random key to be used for the main, symmetric
  algorithm.

  \begin{itemize}
  \itemsep1pt\parskip0pt\parsep0pt
  \item
    It encrypts it using an algorithm also agreed upon during the Hello
    phase, and the server's public key (found on its SSL certificate).
  \item
    It sends this encrypted key to the server, where it is decrypted
    using the server's private key, and the interesting parts of the
    \href{http://www.dictionary.com/browse/handshake}{handshake} are
    complete.
  \end{itemize}
\item
  ¿Que tipo de cifrado se utiliza una vez que a finalizado con éxito la
  fase de handshake?
\item
  The parties are sufficiently happy that they are talking to the right
  person, and have secretly agreed on a key to symmetrically encrypt the
  data that they are about to send each other.
\item
  ¿Cuales son las dos razones por las que podríamos confiar en un
  certificado SSL?
\item
  There are 2 sensible reasons why you might trust a certificate:

  \begin{itemize}
  \itemsep1pt\parskip0pt\parsep0pt
  \item
    If it's on a list of certificates that you trust implicitly
  \item
    If it's able to prove that it is trusted by the controller of one of
    the certificates on the above list
  \item
    The first criteria is easy to check. Your browser has a
    pre-installed list of trusted SSL certificates from Certificate
    Authorities (CAs) that you can view, add and remove from.
  \item
    These certificates are controlled by a centralised group of (in
    theory, and generally in practice) extremely secure, reliable and
    trustworthy organisations like

    \begin{itemize}
    \itemsep1pt\parskip0pt\parsep0pt
    \item
      \href{https://letsencrypt.org/}{Let's Encrypt} (Let's Encrypt is a
      free, automated, and open Certificate Authority),
    \item
      \href{http://www.cacert.org/}{CAcert.org es una autoridad
      certificadora dirigida por la comunidad que emite certificados
      gratuitos al público}
    \item
      Symantec,
    \item
      Comodo and
    \item
      GoDaddy.
    \end{itemize}
  \end{itemize}
\item
  Describa como funciona una firma digital
\item
  As already noted, SSL certificates have an associated public/private
  key pair

  \begin{itemize}
  \itemsep1pt\parskip0pt\parsep0pt
  \item
    The public key is distributed as part of the certificate, and the
    private key is kept incredibly safely guarded
  \item
    This pair of asymmetric keys is used in the SSL
    \href{http://www.dictionary.com/browse/handshake}{handshake} to
    exchange a further key for both parties to symmetrically encrypt and
    decrypt data
  \item
    \textbf{The client uses the server's public key to encrypt the
    symmetric key and send it securely to the server, and the server
    uses its private key to decrypt it}
  \item
    \includegraphics{https://raviranjankr.files.wordpress.com/2012/08/asymmetric-encryption.gif}
  \item
    Anyone can encrypt using the public key, but only the server can
    decrypt using the private key
  \end{itemize}
\item
  The opposite is true for a digital signature.

  \begin{itemize}
  \itemsep1pt\parskip0pt\parsep0pt
  \item
    A certificate can be \emph{``signed''} by another authority,
    \href{https://www.google.es/webhp?sourceid=chrome-instant\&ion=1\&espv=2\&ie=UTF-8\#q=define\%20whereby}{whereby}
    the authority effectively goes on record as saying
  \end{itemize}

  \emph{``We have verified that the controller of this certificate also
  controls the property (domain) listed on the certificate''}.

  \begin{itemize}
  \itemsep1pt\parskip0pt\parsep0pt
  \item
    In this case the authority uses their private key to (broadly
    speaking) encrypt the contents of the certificate, and this cipher
    text is attached to the certificate as its digital signature.
  \item
    Anyone can decrypt this signature using the authority's public key,
    and verify that it results in the expected decrypted value.
  \item
    But only the authority can encrypt content using the private key,
    and so only the authority can actually create a valid signature in
    the first place.
  \end{itemize}
\item
  So if a server comes along claiming to have a certificate for
  Microsoft.com that is signed by Symantec (or some other CA), your
  browser doesn't have to take its word for it.

  \begin{itemize}
  \itemsep1pt\parskip0pt\parsep0pt
  \item
    If it is legit, Symantec will have used their (ultra-secret) private
    key to generate the server's SSL certificate's digital signature,
    and so your browser use can use their (ultra-public) public key to
    check that this signature is valid.
  \item
    Symantec will have taken steps to ensure the organisation they are
    signing for really does own Microsoft.com, and so given that your
    client trusts Symantec, it can be sure that it really is talking to
    Microsoft Inc.
    \includegraphics{http://www.hill2dot0.com/wiki/images/f/ff/Digital_Signature.jpg}
  \end{itemize}
\item
  Pueden en un coffee shop conocer los contenidos de mi tráfico HTTPS
  desde mi portátil sobre su red?
\item
  Nope.

  \begin{itemize}
  \itemsep1pt\parskip0pt\parsep0pt
  \item
    The magic of public-key cryptography means that an attacker can
    watch every single byte of data exchanged between your client and
    the server and still have no idea what you are saying to each other
    beyond roughly how much data you are exchanging.
  \item
    However, your normal HTTP traffic is still very vulnerable on an
    insecure wi-fi network, and a flimsy website can fall victim to any
    number of workarounds that somehow trick you into sending HTTPS
    traffic either over plain HTTP or just to the wrong place
    completely.
  \item
    For example, even if a login form submits a username/password combo
    over HTTPS, if the form itself is loaded insecurely over HTTP then
    an attacker could intercept the form's HTML on its way to your
    machine and modify it to send the login details to their own
    endpoint.
  \end{itemize}
\item
  Puede mi empresa conocer los contenidos de mi tráfico HTTPS sobre la
  red cuando uso la máquina que me proveen?
\item
  If you are also using a machine controlled by your company, then yes.

  \begin{itemize}
  \itemsep1pt\parskip0pt\parsep0pt
  \item
    Remember that at the root of every chain of trust lies an implicitly
    trusted CA, and that a list of these authorities is stored in your
    browser.
  \item
    Your company could use their access to your machine to \textbf{add
    their own self-signed certificate to this list of CAs}.
  \item
    They could then intercept all of your HTTPS requests, presenting
    certificates claiming to represent the appropriate website, signed
    by their fake-CA and therefore unquestioningly trusted by your
    browser.
  \item
    Since you would be encrypting all of your HTTPS requests using their
    dodgy certificate's public key, they could use the corresponding
    private key to decrypt and inspect (even modify) your request, and
    then send it onto it's intended location.
  \item
    They probably don't. But they could.
  \end{itemize}
\item
  Incidentally, this is also how you use a proxy to inspect and modify
  the otherwise inaccessible
  \href{http://nickfishman.com/post/50557873036/reverse-engineering-native-apps-by-intercepting-network}{HTTPS
  requests made by an iPhone app}.
\end{itemize}
