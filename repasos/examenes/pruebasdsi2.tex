\subsection{Preguntas de Repaso de Pruebas con Mocha, Chai y
Should}\label{preguntas-de-repaso-de-pruebas-con-mocha-chai-y-should}

\begin{enumerate}
\def\labelenumi{\arabic{enumi}.}
\itemsep1pt\parskip0pt\parsep0pt
\item
  ¿Como creamos el directorio con el esqueleto inicial para las pruebas
  con mocha? 
\item
  En este ejemplo se usa Chai assert. Rellene lo que falta en estas
  pruebas del código del conversor de temperatura:
\end{enumerate}

\begin{verbatim}
var assert = chai.______;

suite('temperature', function() {
    test('[1,{a: 2}] == [1,{a: 2}]', function() {
      assert._________([1, {a:2}], [1, {a:2}]);
    });
    test('5X = error', function() {
        original.value = "5X";
        calculate();
        assert._____(converted.innerHTML, /ERROR/);
    });
});
\end{verbatim}

\begin{enumerate}
\def\labelenumi{\arabic{enumi}.}
\setcounter{enumi}{2}
\itemsep1pt\parskip0pt\parsep0pt
\item
  Este es un fichero \texttt{test/index.html} apto para ejecutar las
  pruebas con Mocha y Chai en la práctica de la Temperatura en un
  navegador. Rellene las partes que faltan

  \begin{itemize}
  \itemsep1pt\parskip0pt\parsep0pt
  \item
    Sugerencias: El id es el usado por mocha para producir su salida de
    las pruebas, es necesario cargar \texttt{chai} y \texttt{mocha} y
    establecer el estilo de pruebas (\texttt{tdd}, \texttt{bdd}, etc.) y
    por útlimo ejecutar \texttt{mocha}
  \end{itemize}
\end{enumerate}

\begin{verbatim}
[~/srcPLgrado/temperature(karma)]$ cat tests/index.html
<!DOCTYPE html>
<html>
  <head>
    <title>Mocha</title>
    <meta http-equiv="Content-Type" content="text/html; charset=UTF-8">
    <meta name="viewport" content="width=device-width, initial-scale=1.0">
    <link rel="stylesheet" href="mocha.css" />
  </head>
  <body>
    <div id="_____"></div>   <!-- para la salida de las pruebas -->
    <input id="original" placeholder="32F" size="50">
    <span class="output" id="converted"></span>

    <script src="________"></script>
    <script src="________"></script>
    <script src="../temperature.js"></script>
    <script>mocha._____('___')</script>
    <script src="tests.js"></script>

    <script>
      mocha.___();
    </script>
  </body>
</html>
\end{verbatim}

\begin{enumerate}
\def\labelenumi{\arabic{enumi}.}
\setcounter{enumi}{3}
\itemsep1pt\parskip0pt\parsep0pt
\item
  Rellene las partes que faltan del fichero con las pruebas TDD en Mocha
  y Chai para la práctica de la temperatura:
\end{enumerate}

\begin{verbatim}
        [~/srcPLgrado/temperature(karma)]$ cat tests/tests.js
        var assert = ____.assert;

        _____('temperature', function() {

            ____('32F = 0C', function() {
                original.value = "32F";
                calculate();
                assert._________(converted.innerHTML, "0.0 Celsius");
            });
        });
\end{verbatim}

\begin{enumerate}
\def\labelenumi{\arabic{enumi}.}
\setcounter{enumi}{4}
\itemsep1pt\parskip0pt\parsep0pt
\item
  ¿Como puedo ejecutar las pruebas escritas usando Mocha y Chai usando
  el comando \texttt{npm test}?. (no se asume el uso de Karma en esta
  pregunta) Explique como hacerlo.
\end{enumerate}

\begin{enumerate}
\def\labelenumi{\arabic{enumi}.}
\setcounter{enumi}{5}
\itemsep1pt\parskip0pt\parsep0pt
\item
  El siguiente ejemplo corresponde al ejemplo de pruebas que vimos para
  la renderización de una tabla correspondiente al capítulo \emph{The
  Secret Life Of Objects} que usa \texttt{mocha}y \texttt{should}.
  Rellena las partes que faltan:
\end{enumerate}

\begin{verbatim}
________("drawTable", function() {
  __("must draw the checkerboard correctly", function() {
    /* There are 5 columns and 5 rows and a white space between columns*/
    drawTable(checkerboard()).should._____(/^(([# ]{2}(\s|$)){5}){5}$/);
  })
});
\end{verbatim}
