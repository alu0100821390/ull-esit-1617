\subsection{Preguntas de Repaso de
Heroku}\label{preguntas-de-repaso-de-heroku}

\begin{enumerate}
\def\labelenumi{\arabic{enumi}.}
\item
  Una vez instalado el Heroku cli nos debemos autenticar en heroku con
  el cliente. ¿Cual es el comando para autenticarnos?
\item
  ¿Cual es el comando para crear nuestra aplicación en Heroku (suponemos
  que ya esta bajo el control de \texttt{git}? ¿Qué remoto tendremos
  definido después de crear nuestra aplicación en Heroku?
\item
  ¿Cual es el comando para desplegar nuestra aplicación en Heroku?
\item
  Si la versión que queremos publicar en heroku no está en la rama
  \texttt{master} sino que está en la rama \texttt{tutu} ¿Como debemos
  modificar el comando anterior?
\item
  ¿Con que comando puedo abrir una ventana en el navegador que visite la
  aplicación desplegada? ¿Que formato tiene la URL para nuestra
  aplicación?
\item
  ¿Con que comando puedo ver los logs de la aplicación desplegada?
\item
  ¿Como se debe llamar el fichero en el que explicito que comando debe
  usarse para arrancar nuestra aplicación en Heroku?
\item
  Heroku se percata que nuestra aplicación es una aplicación
  desarrollada con \texttt{Node.js} por la presencia de un cierto
  fichero. ¿De que fichero estamos hablando?
\item
  ¿Cual es la mejor forma de ejecutar en local una aplicación express.js
  que va a ser desplegada en Heroku?
\item
  Explique los pasos para desplegar una aplicación en Heroku
\item
  Explique como resolver los problemas que pueden surgir cuando la
  aplicación desplegada en Heroku no funciona correctamente
\item
  \href{../recursos/heroku.md}{¿Como consulto el token para hacer uso de
  la API de Heroku?}
\item
  \href{../recursos/heroku.md}{¿Cómo creo una app en Heroku usando la
  API de Heroku?}
\end{enumerate}

\begin{itemize}
\item
  \href{https://devcenter.heroku.com/articles/setting-up-apps-using-the-heroku-platform-api\#creating-an-app-setup}{Véase}
\item
  With a source tarball, which contains an app.json, call the
  https://api.heroku.com/app-setups endpoint to setup the app.json
  enabled application on Heroku. The request body must contain a source
  URL that points to the tarball of your application's source code.
\item
  Let's use cURL to call the app-setups endpoint:

\begin{verbatim}
$ curl -n -X POST https://api.heroku.com/app-setups \
-H "Content-Type:application/json" \
-H "Accept:application/vnd.heroku+json; version=3" \
-d '{"source_blob": { "url":"https://github.com/heroku/ruby-rails-sample/tarball/master/"} }'
\end{verbatim}
\item
  Explique los pasos para publicar un libro GitBook en Heroku
\end{itemize}
